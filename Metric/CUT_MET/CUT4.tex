\documentclass[12pt]{article}
\usepackage{graphicx,amssymb,amsmath,vmargin,multicol}
\usepackage[english]{babel}
\title{The Cone CUT4}
\setpapersize{custom}{21cm}{29.7cm}
\setmarginsrb{1cm}{1cm}{1cm}{2cm}{0pt}{0pt}{0pt}{0pt}
\begin{document}
\maketitle
\section{The intrisic informations concerning the graph of extreme rays of CUT4.ext}
The graph considered is made of $7$ elements.\\
The group acting on CUT4.ext is ReprS4\\
Under the group action we have in fact only $2$ orbits to consider.\\
The diameter of the graph is $1$.\\
The list of orbits is with adjacency and Size info:
\begin{center}
\scriptsize
\begin{tabular}{ccccccc|c|c}
E1:&0&0&1&0&1&1&6&4\\
E2:&0&1&1&1&1&0&6&3\\
\end{tabular}
\end{center}
The representation matrix is:
\begin{center}
\scriptsize
\begin{tabular}{|c|cc|c|c|}
\hline
&E1&E2&Adj.&Size\\
\hline
E1& 3& 3&6&4\\
E2& 4& 2&6&3\\
\hline
Size&4&3&&7\\
\hline
\end{tabular}
\end{center}
\subsection{The local graphs}
The complemented local graph are:\\
We consider graph with cardinal inferior or equal to $20$\\
The complemented local graph for orbit $1$ is:
\begin{equation*}
\begin{array}{rrcl}
1&0&:&\\
2&0&:&\\
3&0&:&\\
4&0&:&\\
5&0&:&\\
6&0&:&\\
\end{array}
\end{equation*}
The complemented local graph for orbit $2$ is:
\begin{equation*}
\begin{array}{rrcl}
1&0&:&\\
2&0&:&\\
3&0&:&\\
4&0&:&\\
5&0&:&\\
6&0&:&\\
\end{array}
\end{equation*}
\subsection{scalar product between elements of orbits}
\noindent This subsection made sense only if the representation is orthogonal which is the usual case
Considering Orbit 1
\begin{enumerate}
\item Orbit Size is $4$.
\item Norm of elements is $3$.
\item Possible scalar product : $1$
\end{enumerate}
Considering Orbit 2
\begin{enumerate}
\item Orbit Size is $3$.
\item Norm of elements is $4$.
\item Possible scalar product : $2$
\end{enumerate}
\subsection{invariant group of Orbits}
\noindent The invariant group of $x$ is the set of $g$ such that $g(x)=x$ denoted by $G_x$.\\
If $x$ and $y$ are in the same orbit then the groups $G_x$ and  $G_y$ are conjuguate.\\
For any element $x$ we denote $O_x$ its Orbit, we have $|O_x|\times |G_x|=|G|$
We print only nontrivial groups
Invariant group of Orbit $1$:
\begin{enumerate}
\item Group Size is $6$.
\item Elements are : (3 2 1 4  ), (3 1 2 4  ), (2 3 1 4  ), (1 3 2 4  ), (2 1 3 4  ), (1 2 3 4  )
\end{enumerate}
Invariant group of Orbit $2$:
\begin{enumerate}
\item Group Size is $8$.
\item Elements are : (4 3 2 1  ), (4 3 1 2  ), (3 4 2 1  ), (3 4 1 2  ), (2 1 4 3  ), (1 2 4 3  ), (2 1 3 4  ), (1 2 3 4  )
\end{enumerate}
\section{The intrisic informations concerning the graph of facet rays of CUT4.ine}
The graph considered is made of $12$ elements.\\
The group acting on CUT4.ine is ReprS4\\
Under the group action we have in fact only $1$ orbits to consider.\\
The diameter of the graph is $2$.\\
The list of orbits is with adjacency and Size info:
\begin{center}
\scriptsize
\begin{tabular}{ccccccc|c|c}
F1:&-1&0&1&0&1&0&5&12\\
\end{tabular}
\end{center}
The representation matrix is:
\begin{center}
\scriptsize
\begin{tabular}{|c|c|c|c|}
\hline
&F1&Adj.&Size\\
\hline
F1& 5&5&12\\
\hline
Size&12&&12\\
\hline
\end{tabular}
\end{center}
\subsection{The local graphs}
The complemented local graph are:\\
We consider graph with cardinal inferior or equal to $20$\\
The complemented local graph for orbit $1$ is:
\begin{equation*}
\begin{array}{rrcl}
1&2&:&\,\,4\,\,5\\
2&2&:&\,\,4\,\,5\\
3&2&:&\,\,4\,\,5\\
4&3&:&\,\,1\,\,2\,\,3\\
5&3&:&\,\,1\,\,2\,\,3\\
\end{array}
\end{equation*}
\subsection{scalar product between elements of orbits}
\noindent This subsection made sense only if the representation is orthogonal which is the usual case
Considering Orbit 1
\begin{enumerate}
\item Orbit Size is $12$.
\item Norm of elements is $3$.
\item Possible scalar product : $-1$, $1$
\end{enumerate}
\subsection{invariant group of Orbits}
\noindent The invariant group of $x$ is the set of $g$ such that $g(x)=x$ denoted by $G_x$.\\
If $x$ and $y$ are in the same orbit then the groups $G_x$ and  $G_y$ are conjuguate.\\
For any element $x$ we denote $O_x$ its Orbit, we have $|O_x|\times |G_x|=|G|$
We print only nontrivial groups
Invariant group of Orbit $1$:
\begin{enumerate}
\item Group Size is $2$.
\item Elements are : (2 1 3 4  ), (1 2 3 4  )
\end{enumerate}
\section{The two incidence matrix}
Incidence between Orbit of Extreme rays and Orbits of Facets
\begin{equation*}
\begin{array}{|c|c|c|}
\hline
&F1&Inc\\
\hline
E1&9&9\\
E2&8&8\\
\hline
\end{array}
\end{equation*}
Incidence between Orbits of Facets and Orbits of Extreme Rays
\begin{equation*}
\begin{array}{|c|cc|c|}
\hline
&E1&E2&Inc\\
\hline
F1&3&2&5\\
\hline
\end{array}
\end{equation*}
\end{document}
