\documentclass[12pt]{article}
\usepackage{graphicx,amssymb,amsmath,vmargin,multicol}
\usepackage[english]{babel}
\title{The Cone CUT6}
\setpapersize{custom}{21cm}{29.7cm}
\setmarginsrb{1cm}{1cm}{1cm}{2cm}{0pt}{0pt}{0pt}{0pt}
\begin{document}
\maketitle
\section{The intrisic informations concerning the graph of extreme rays of CUT6.ext}
The graph considered is made of $31$ elements.\\
The group acting on CUT6.ext is ReprS6\\
Under the group action we have in fact only $3$ orbits to consider.\\
The diameter of the graph is $1$.\\
The list of orbits is with adjacency and Size info:
\begin{center}
\scriptsize
\begin{tabular}{cccccccccccccccc|c|c}
E1:&0&0&0&1&1&0&0&1&1&0&1&1&1&1&0&30&15\\
E2:&0&0&1&1&1&0&1&1&1&1&1&1&0&0&0&30&10\\
E3:&0&0&0&0&1&0&0&0&1&0&0&1&0&1&1&30&6\\
\end{tabular}
\end{center}
The representation matrix is:
\begin{center}
\scriptsize
\begin{tabular}{|c|ccc|c|c|}
\hline
&E1&E2&E3&Adj.&Size\\
\hline
E1& 14& 10& 6&30&15\\
E2& 15& 9& 6&30&10\\
E3& 15& 10& 5&30&6\\
\hline
Size&15&10&6&&31\\
\hline
\end{tabular}
\end{center}
No complemented local graph of size inferior to $20$.
\subsection{scalar product between elements of orbits}
\noindent This subsection made sense only if the representation is orthogonal which is the usual case
Considering Orbit 1
\begin{enumerate}
\item Orbit Size is $15$.
\item Norm of elements is $8$.
\item Possible scalar product : $4$
\end{enumerate}
Considering Orbit 2
\begin{enumerate}
\item Orbit Size is $10$.
\item Norm of elements is $9$.
\item Possible scalar product : $5$
\end{enumerate}
Considering Orbit 3
\begin{enumerate}
\item Orbit Size is $6$.
\item Norm of elements is $5$.
\item Possible scalar product : $1$
\end{enumerate}
\subsection{invariant group of Orbits}
\noindent The invariant group of $x$ is the set of $g$ such that $g(x)=x$ denoted by $G_x$.\\
If $x$ and $y$ are in the same orbit then the groups $G_x$ and  $G_y$ are conjuguate.\\
For any element $x$ we denote $O_x$ its Orbit, we have $|O_x|\times |G_x|=|G|$
We print only nontrivial groups
Invariant group of Orbit $1$:
\begin{enumerate}
\item Group Size is $48$.
\item Elements are : (4 3 2 1 6 5  ), (4 3 1 2 6 5  ), (4 2 3 1 6 5  ), (4 1 3 2 6 5  ), (4 2 1 3 6 5  ), (4 1 2 3 6 5  ), (3 4 2 1 6 5  ), (3 4 1 2 6 5  ), (2 4 3 1 6 5  ), (1 4 3 2 6 5  ), (2 4 1 3 6 5  ), (1 4 2 3 6 5  ), (3 2 4 1 6 5  ), (3 1 4 2 6 5  ), (2 3 4 1 6 5  ), (1 3 4 2 6 5  ), (2 1 4 3 6 5  ), (1 2 4 3 6 5  ), (3 2 1 4 6 5  ), (3 1 2 4 6 5  ), (2 3 1 4 6 5  ), (1 3 2 4 6 5  ), (2 1 3 4 6 5  ), (1 2 3 4 6 5  ), (4 3 2 1 5 6  ), (4 3 1 2 5 6  ), (4 2 3 1 5 6  ), (4 1 3 2 5 6  ), (4 2 1 3 5 6  ), (4 1 2 3 5 6  ), (3 4 2 1 5 6  ), (3 4 1 2 5 6  ), (2 4 3 1 5 6  ), (1 4 3 2 5 6  ), (2 4 1 3 5 6  ), (1 4 2 3 5 6  ), (3 2 4 1 5 6  ), (3 1 4 2 5 6  ), (2 3 4 1 5 6  ), (1 3 4 2 5 6  ), (2 1 4 3 5 6  ), (1 2 4 3 5 6  ), (3 2 1 4 5 6  ), (3 1 2 4 5 6  ), (2 3 1 4 5 6  ), (1 3 2 4 5 6  ), (2 1 3 4 5 6  ), (1 2 3 4 5 6  )
\end{enumerate}
Invariant group of Orbit $2$:
\begin{enumerate}
\item Group Size is $72$.
\item Elements are : (6 5 4 3 2 1  ), (6 5 4 3 1 2  ), (6 5 4 2 3 1  ), (6 5 4 1 3 2  ), (6 5 4 2 1 3  ), (6 5 4 1 2 3  ), (6 4 5 3 2 1  ), (6 4 5 3 1 2  ), (6 4 5 2 3 1  ), (6 4 5 1 3 2  ), (6 4 5 2 1 3  ), (6 4 5 1 2 3  ), (5 6 4 3 2 1  ), (5 6 4 3 1 2  ), (5 6 4 2 3 1  ), (5 6 4 1 3 2  ), (5 6 4 2 1 3  ), (5 6 4 1 2 3  ), (4 6 5 3 2 1  ), (4 6 5 3 1 2  ), (4 6 5 2 3 1  ), (4 6 5 1 3 2  ), (4 6 5 2 1 3  ), (4 6 5 1 2 3  ), (5 4 6 3 2 1  ), (5 4 6 3 1 2  ), (5 4 6 2 3 1  ), (5 4 6 1 3 2  ), (5 4 6 2 1 3  ), (5 4 6 1 2 3  ), (4 5 6 3 2 1  ), (4 5 6 3 1 2  ), (4 5 6 2 3 1  ), (4 5 6 1 3 2  ), (4 5 6 2 1 3  ), (4 5 6 1 2 3  ), (3 2 1 6 5 4  ), (3 1 2 6 5 4  ), (2 3 1 6 5 4  ), (1 3 2 6 5 4  ), (2 1 3 6 5 4  ), (1 2 3 6 5 4  ), (3 2 1 6 4 5  ), (3 1 2 6 4 5  ), (2 3 1 6 4 5  ), (1 3 2 6 4 5  ), (2 1 3 6 4 5  ), (1 2 3 6 4 5  ), (3 2 1 5 6 4  ), (3 1 2 5 6 4  ), (2 3 1 5 6 4  ), (1 3 2 5 6 4  ), (2 1 3 5 6 4  ), (1 2 3 5 6 4  ), (3 2 1 4 6 5  ), (3 1 2 4 6 5  ), (2 3 1 4 6 5  ), (1 3 2 4 6 5  ), (2 1 3 4 6 5  ), (1 2 3 4 6 5  ), (3 2 1 5 4 6  ), (3 1 2 5 4 6  ), (2 3 1 5 4 6  ), (1 3 2 5 4 6  ), (2 1 3 5 4 6  ), (1 2 3 5 4 6  ), (3 2 1 4 5 6  ), (3 1 2 4 5 6  ), (2 3 1 4 5 6  ), (1 3 2 4 5 6  ), (2 1 3 4 5 6  ), (1 2 3 4 5 6  )
\end{enumerate}
Invariant group of Orbit $3$:
\begin{enumerate}
\item Group Size is $120$.
\item Elements are : (5 4 3 2 1 6  ), (5 4 3 1 2 6  ), (5 4 2 3 1 6  ), (5 4 1 3 2 6  ), (5 4 2 1 3 6  ), (5 4 1 2 3 6  ), (5 3 4 2 1 6  ), (5 3 4 1 2 6  ), (5 2 4 3 1 6  ), (5 1 4 3 2 6  ), (5 2 4 1 3 6  ), (5 1 4 2 3 6  ), (5 3 2 4 1 6  ), (5 3 1 4 2 6  ), (5 2 3 4 1 6  ), (5 1 3 4 2 6  ), (5 2 1 4 3 6  ), (5 1 2 4 3 6  ), (5 3 2 1 4 6  ), (5 3 1 2 4 6  ), (5 2 3 1 4 6  ), (5 1 3 2 4 6  ), (5 2 1 3 4 6  ), (5 1 2 3 4 6  ), (4 5 3 2 1 6  ), (4 5 3 1 2 6  ), (4 5 2 3 1 6  ), (4 5 1 3 2 6  ), (4 5 2 1 3 6  ), (4 5 1 2 3 6  ), (3 5 4 2 1 6  ), (3 5 4 1 2 6  ), (2 5 4 3 1 6  ), (1 5 4 3 2 6  ), (2 5 4 1 3 6  ), (1 5 4 2 3 6  ), (3 5 2 4 1 6  ), (3 5 1 4 2 6  ), (2 5 3 4 1 6  ), (1 5 3 4 2 6  ), (2 5 1 4 3 6  ), (1 5 2 4 3 6  ), (3 5 2 1 4 6  ), (3 5 1 2 4 6  ), (2 5 3 1 4 6  ), (1 5 3 2 4 6  ), (2 5 1 3 4 6  ), (1 5 2 3 4 6  ), (4 3 5 2 1 6  ), (4 3 5 1 2 6  ), (4 2 5 3 1 6  ), (4 1 5 3 2 6  ), (4 2 5 1 3 6  ), (4 1 5 2 3 6  ), (3 4 5 2 1 6  ), (3 4 5 1 2 6  ), (2 4 5 3 1 6  ), (1 4 5 3 2 6  ), (2 4 5 1 3 6  ), (1 4 5 2 3 6  ), (3 2 5 4 1 6  ), (3 1 5 4 2 6  ), (2 3 5 4 1 6  ), (1 3 5 4 2 6  ), (2 1 5 4 3 6  ), (1 2 5 4 3 6  ), (3 2 5 1 4 6  ), (3 1 5 2 4 6  ), (2 3 5 1 4 6  ), (1 3 5 2 4 6  ), (2 1 5 3 4 6  ), (1 2 5 3 4 6  ), (4 3 2 5 1 6  ), (4 3 1 5 2 6  ), (4 2 3 5 1 6  ), (4 1 3 5 2 6  ), (4 2 1 5 3 6  ), (4 1 2 5 3 6  ), (3 4 2 5 1 6  ), (3 4 1 5 2 6  ), (2 4 3 5 1 6  ), (1 4 3 5 2 6  ), (2 4 1 5 3 6  ), (1 4 2 5 3 6  ), (3 2 4 5 1 6  ), (3 1 4 5 2 6  ), (2 3 4 5 1 6  ), (1 3 4 5 2 6  ), (2 1 4 5 3 6  ), (1 2 4 5 3 6  ), (3 2 1 5 4 6  ), (3 1 2 5 4 6  ), (2 3 1 5 4 6  ), (1 3 2 5 4 6  ), (2 1 3 5 4 6  ), (1 2 3 5 4 6  ), (4 3 2 1 5 6  ), (4 3 1 2 5 6  ), (4 2 3 1 5 6  ), (4 1 3 2 5 6  ), (4 2 1 3 5 6  ), (4 1 2 3 5 6  ), (3 4 2 1 5 6  ), (3 4 1 2 5 6  ), (2 4 3 1 5 6  ), (1 4 3 2 5 6  ), (2 4 1 3 5 6  ), (1 4 2 3 5 6  ), (3 2 4 1 5 6  ), (3 1 4 2 5 6  ), (2 3 4 1 5 6  ), (1 3 4 2 5 6  ), (2 1 4 3 5 6  ), (1 2 4 3 5 6  ), (3 2 1 4 5 6  ), (3 1 2 4 5 6  ), (2 3 1 4 5 6  ), (1 3 2 4 5 6  ), (2 1 3 4 5 6  ), (1 2 3 4 5 6  )
\end{enumerate}
\section{The intrisic informations concerning the graph of facet rays of CUT6.ine}
The graph considered is made of $210$ elements.\\
The group acting on CUT6.ine is ReprS6\\
Under the group action we have in fact only $4$ orbits to consider.\\
The diameter of the graph is $3$.\\
The list of orbits is with adjacency and Size info:
\begin{center}
\scriptsize
\begin{tabular}{cccccccccccccccc|c|c}
F1:&-1&0&0&0&1&0&0&0&1&0&0&0&0&0&0&98&60\\
F2:&-1&-1&0&1&1&-1&0&1&1&0&1&1&0&0&-1&54&60\\
F3:&-1&-1&1&1&2&-1&1&1&2&1&1&2&-1&-2&-2&14&60\\
F4:&-1&-1&-1&1&2&-1&-1&1&2&-1&1&2&1&2&-2&14&30\\
\end{tabular}
\end{center}
The representation matrix is:
\begin{center}
\scriptsize
\begin{tabular}{|c|cccc|c|c|}
\hline
&F1&F2&F3&F4&Adj.&Size\\
\hline
F1& 45& 39& 9& 5&98&60\\
F2& 39& 8& 5& 2&54&60\\
F3& 9& 5& 0& 0&14&60\\
F4& 10& 4& 0& 0&14&30\\
\hline
Size&60&60&60&30&&210\\
\hline
\end{tabular}
\end{center}
\subsection{The local graphs}
The complemented local graph are:\\
We consider graph with cardinal inferior or equal to $20$\\
The complemented local graph for orbit $3$ is:
\begin{equation*}
\begin{array}{rrcl}
1&0&:&\\
2&0&:&\\
3&0&:&\\
4&0&:&\\
5&0&:&\\
6&0&:&\\
7&0&:&\\
8&0&:&\\
9&0&:&\\
10&0&:&\\
11&0&:&\\
12&0&:&\\
13&0&:&\\
14&0&:&\\
\end{array}
\end{equation*}
The complemented local graph for orbit $4$ is:
\begin{equation*}
\begin{array}{rrcl}
1&0&:&\\
2&0&:&\\
3&0&:&\\
4&0&:&\\
5&0&:&\\
6&0&:&\\
7&0&:&\\
8&0&:&\\
9&0&:&\\
10&0&:&\\
11&0&:&\\
12&0&:&\\
13&0&:&\\
14&0&:&\\
\end{array}
\end{equation*}
\subsection{scalar product between elements of orbits}
\noindent This subsection made sense only if the representation is orthogonal which is the usual case
Considering Orbit 1
\begin{enumerate}
\item Orbit Size is $60$.
\item Norm of elements is $3$.
\item Possible scalar product : $-1$, $0$, $1$
\end{enumerate}
Considering Orbit 2
\begin{enumerate}
\item Orbit Size is $60$.
\item Norm of elements is $10$.
\item Possible scalar product : $-2$, $0$, $2$, $6$
\end{enumerate}
Considering Orbit 3
\begin{enumerate}
\item Orbit Size is $60$.
\item Norm of elements is $30$.
\item Possible scalar product : $-10$, $-6$, $-4$, $2$, $26$
\end{enumerate}
Considering Orbit 4
\begin{enumerate}
\item Orbit Size is $30$.
\item Norm of elements is $30$.
\item Possible scalar product : $-6$, $-4$, $2$, $26$
\end{enumerate}
\subsection{invariant group of Orbits}
\noindent The invariant group of $x$ is the set of $g$ such that $g(x)=x$ denoted by $G_x$.\\
If $x$ and $y$ are in the same orbit then the groups $G_x$ and  $G_y$ are conjuguate.\\
For any element $x$ we denote $O_x$ its Orbit, we have $|O_x|\times |G_x|=|G|$
We print only nontrivial groups
Invariant group of Orbit $1$:
\begin{enumerate}
\item Group Size is $12$.
\item Elements are : (2 1 5 4 3 6  ), (1 2 5 4 3 6  ), (2 1 5 3 4 6  ), (1 2 5 3 4 6  ), (2 1 4 5 3 6  ), (1 2 4 5 3 6  ), (2 1 3 5 4 6  ), (1 2 3 5 4 6  ), (2 1 4 3 5 6  ), (1 2 4 3 5 6  ), (2 1 3 4 5 6  ), (1 2 3 4 5 6  )
\end{enumerate}
Invariant group of Orbit $2$:
\begin{enumerate}
\item Group Size is $12$.
\item Elements are : (3 2 1 4 6 5  ), (3 1 2 4 6 5  ), (2 3 1 4 6 5  ), (1 3 2 4 6 5  ), (2 1 3 4 6 5  ), (1 2 3 4 6 5  ), (3 2 1 4 5 6  ), (3 1 2 4 5 6  ), (2 3 1 4 5 6  ), (1 3 2 4 5 6  ), (2 1 3 4 5 6  ), (1 2 3 4 5 6  )
\end{enumerate}
Invariant group of Orbit $3$:
\begin{enumerate}
\item Group Size is $12$.
\item Elements are : (3 2 1 5 4 6  ), (3 1 2 5 4 6  ), (2 3 1 5 4 6  ), (1 3 2 5 4 6  ), (2 1 3 5 4 6  ), (1 2 3 5 4 6  ), (3 2 1 4 5 6  ), (3 1 2 4 5 6  ), (2 3 1 4 5 6  ), (1 3 2 4 5 6  ), (2 1 3 4 5 6  ), (1 2 3 4 5 6  )
\end{enumerate}
Invariant group of Orbit $4$:
\begin{enumerate}
\item Group Size is $24$.
\item Elements are : (4 3 2 1 5 6  ), (4 3 1 2 5 6  ), (4 2 3 1 5 6  ), (4 1 3 2 5 6  ), (4 2 1 3 5 6  ), (4 1 2 3 5 6  ), (3 4 2 1 5 6  ), (3 4 1 2 5 6  ), (2 4 3 1 5 6  ), (1 4 3 2 5 6  ), (2 4 1 3 5 6  ), (1 4 2 3 5 6  ), (3 2 4 1 5 6  ), (3 1 4 2 5 6  ), (2 3 4 1 5 6  ), (1 3 4 2 5 6  ), (2 1 4 3 5 6  ), (1 2 4 3 5 6  ), (3 2 1 4 5 6  ), (3 1 2 4 5 6  ), (2 3 1 4 5 6  ), (1 3 2 4 5 6  ), (2 1 3 4 5 6  ), (1 2 3 4 5 6  )
\end{enumerate}
\section{The two incidence matrix}
Incidence between Orbit of Extreme rays and Orbits of Facets
\begin{equation*}
\begin{array}{|c|cccc|c|}
\hline
&F1&F2&F3&F4&Inc\\
\hline
E1&44&36&36&8&124\\
E2&42&36&18&18&114\\
E3&50&40&20&20&130\\
\hline
\end{array}
\end{equation*}
Incidence between Orbits of Facets and Orbits of Extreme Rays
\begin{equation*}
\begin{array}{|c|ccc|c|}
\hline
&E1&E2&E3&Inc\\
\hline
F1&11&7&5&23\\
F2&9&6&4&19\\
F3&9&3&2&14\\
F4&4&6&4&14\\
\hline
\end{array}
\end{equation*}
\end{document}
