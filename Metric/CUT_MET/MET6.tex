\documentclass[12pt]{article}
\usepackage{graphicx,amssymb,amsmath,vmargin,multicol}
\usepackage[english]{babel}
\title{The Cone MET6}
\setpapersize{custom}{21cm}{29.7cm}
\setmarginsrb{1cm}{1cm}{1cm}{2cm}{0pt}{0pt}{0pt}{0pt}
\begin{document}
\maketitle
\section{The intrisic informations concerning the graph of extreme rays of MET6.ext}
The graph considered is made of $296$ elements.\\
The group acting on MET6.ext is ReprS6\\
Under the group action we have in fact only $8$ orbits to consider.\\
The diameter of the graph is $2$.\\
The list of orbits is with adjacency and Size info:
\begin{center}
\scriptsize
\begin{tabular}{cccccccccccccccc|c|c}
E1:&0&0&0&0&1&0&0&0&1&0&0&1&0&1&1&190&6\\
E2:&0&0&0&1&1&0&0&1&1&0&1&1&1&1&0&158&15\\
E3:&0&0&1&1&1&0&1&1&1&1&1&1&0&0&0&156&10\\
E4:&1&1&1&2&2&2&2&1&1&2&1&1&1&1&2&57&10\\
E5:&0&1&1&1&2&1&1&1&2&2&2&1&2&1&1&34&60\\
E6:&1&1&1&1&2&2&2&2&1&2&2&1&2&1&1&32&15\\
E7:&0&1&1&2&2&1&1&2&2&2&1&1&1&1&2&30&90\\
E8:&1&1&1&2&2&2&2&1&1&2&1&1&1&3&2&23&90\\
\end{tabular}
\end{center}
The representation matrix is:
\begin{center}
\scriptsize
\begin{tabular}{|c|cccccccc|c|c|}
\hline
&E1&E2&E3&E4&E5&E6&E7&E8&Adj.&Size\\
\hline
E1& 5& 15& 10& 10& 50& 10& 60& 30&190&6\\
E2& 6& 14& 10& 6& 36& 8& 30& 48&158&15\\
E3& 6& 15& 9& 9& 18& 9& 54& 36&156&10\\
E4& 6& 9& 9& 0& 6& 0& 18& 9&57&10\\
E5& 5& 9& 3& 1& 2& 2& 6& 6&34&60\\
E6& 4& 8& 6& 0& 8& 0& 6& 0&32&15\\
E7& 4& 5& 6& 2& 4& 1& 4& 4&30&90\\
E8& 2& 8& 4& 1& 4& 0& 4& 0&23&90\\
\hline
Size&6&15&10&10&60&15&90&90&&296\\
\hline
\end{tabular}
\end{center}
No complemented local graph of size inferior to $20$.
\subsection{scalar product between elements of orbits}
\noindent This subsection made sense only if the representation is orthogonal which is the usual case
Considering Orbit 1
\begin{enumerate}
\item Orbit Size is $6$.
\item Norm of elements is $5$.
\item Possible scalar product : $1$
\end{enumerate}
Considering Orbit 2
\begin{enumerate}
\item Orbit Size is $15$.
\item Norm of elements is $8$.
\item Possible scalar product : $4$
\end{enumerate}
Considering Orbit 3
\begin{enumerate}
\item Orbit Size is $10$.
\item Norm of elements is $9$.
\item Possible scalar product : $5$
\end{enumerate}
Considering Orbit 4
\begin{enumerate}
\item Orbit Size is $10$.
\item Norm of elements is $33$.
\item Possible scalar product : $29$
\end{enumerate}
Considering Orbit 5
\begin{enumerate}
\item Orbit Size is $60$.
\item Norm of elements is $29$.
\item Possible scalar product : $21$, $23$, $25$
\end{enumerate}
Considering Orbit 6
\begin{enumerate}
\item Orbit Size is $15$.
\item Norm of elements is $36$.
\item Possible scalar product : $32$
\end{enumerate}
Considering Orbit 7
\begin{enumerate}
\item Orbit Size is $90$.
\item Norm of elements is $32$.
\item Possible scalar product : $24$, $26$, $28$
\end{enumerate}
Considering Orbit 8
\begin{enumerate}
\item Orbit Size is $90$.
\item Norm of elements is $41$.
\item Possible scalar product : $33$, $35$, $37$
\end{enumerate}
\subsection{invariant group of Orbits}
\noindent The invariant group of $x$ is the set of $g$ such that $g(x)=x$ denoted by $G_x$.\\
If $x$ and $y$ are in the same orbit then the groups $G_x$ and  $G_y$ are conjuguate.\\
For any element $x$ we denote $O_x$ its Orbit, we have $|O_x|\times |G_x|=|G|$
We print only nontrivial groups
Invariant group of Orbit $1$:
\begin{enumerate}
\item Group Size is $120$.
\item Elements are : (5 4 3 2 1 6  ), (5 4 3 1 2 6  ), (5 4 2 3 1 6  ), (5 4 1 3 2 6  ), (5 4 2 1 3 6  ), (5 4 1 2 3 6  ), (5 3 4 2 1 6  ), (5 3 4 1 2 6  ), (5 2 4 3 1 6  ), (5 1 4 3 2 6  ), (5 2 4 1 3 6  ), (5 1 4 2 3 6  ), (5 3 2 4 1 6  ), (5 3 1 4 2 6  ), (5 2 3 4 1 6  ), (5 1 3 4 2 6  ), (5 2 1 4 3 6  ), (5 1 2 4 3 6  ), (5 3 2 1 4 6  ), (5 3 1 2 4 6  ), (5 2 3 1 4 6  ), (5 1 3 2 4 6  ), (5 2 1 3 4 6  ), (5 1 2 3 4 6  ), (4 5 3 2 1 6  ), (4 5 3 1 2 6  ), (4 5 2 3 1 6  ), (4 5 1 3 2 6  ), (4 5 2 1 3 6  ), (4 5 1 2 3 6  ), (3 5 4 2 1 6  ), (3 5 4 1 2 6  ), (2 5 4 3 1 6  ), (1 5 4 3 2 6  ), (2 5 4 1 3 6  ), (1 5 4 2 3 6  ), (3 5 2 4 1 6  ), (3 5 1 4 2 6  ), (2 5 3 4 1 6  ), (1 5 3 4 2 6  ), (2 5 1 4 3 6  ), (1 5 2 4 3 6  ), (3 5 2 1 4 6  ), (3 5 1 2 4 6  ), (2 5 3 1 4 6  ), (1 5 3 2 4 6  ), (2 5 1 3 4 6  ), (1 5 2 3 4 6  ), (4 3 5 2 1 6  ), (4 3 5 1 2 6  ), (4 2 5 3 1 6  ), (4 1 5 3 2 6  ), (4 2 5 1 3 6  ), (4 1 5 2 3 6  ), (3 4 5 2 1 6  ), (3 4 5 1 2 6  ), (2 4 5 3 1 6  ), (1 4 5 3 2 6  ), (2 4 5 1 3 6  ), (1 4 5 2 3 6  ), (3 2 5 4 1 6  ), (3 1 5 4 2 6  ), (2 3 5 4 1 6  ), (1 3 5 4 2 6  ), (2 1 5 4 3 6  ), (1 2 5 4 3 6  ), (3 2 5 1 4 6  ), (3 1 5 2 4 6  ), (2 3 5 1 4 6  ), (1 3 5 2 4 6  ), (2 1 5 3 4 6  ), (1 2 5 3 4 6  ), (4 3 2 5 1 6  ), (4 3 1 5 2 6  ), (4 2 3 5 1 6  ), (4 1 3 5 2 6  ), (4 2 1 5 3 6  ), (4 1 2 5 3 6  ), (3 4 2 5 1 6  ), (3 4 1 5 2 6  ), (2 4 3 5 1 6  ), (1 4 3 5 2 6  ), (2 4 1 5 3 6  ), (1 4 2 5 3 6  ), (3 2 4 5 1 6  ), (3 1 4 5 2 6  ), (2 3 4 5 1 6  ), (1 3 4 5 2 6  ), (2 1 4 5 3 6  ), (1 2 4 5 3 6  ), (3 2 1 5 4 6  ), (3 1 2 5 4 6  ), (2 3 1 5 4 6  ), (1 3 2 5 4 6  ), (2 1 3 5 4 6  ), (1 2 3 5 4 6  ), (4 3 2 1 5 6  ), (4 3 1 2 5 6  ), (4 2 3 1 5 6  ), (4 1 3 2 5 6  ), (4 2 1 3 5 6  ), (4 1 2 3 5 6  ), (3 4 2 1 5 6  ), (3 4 1 2 5 6  ), (2 4 3 1 5 6  ), (1 4 3 2 5 6  ), (2 4 1 3 5 6  ), (1 4 2 3 5 6  ), (3 2 4 1 5 6  ), (3 1 4 2 5 6  ), (2 3 4 1 5 6  ), (1 3 4 2 5 6  ), (2 1 4 3 5 6  ), (1 2 4 3 5 6  ), (3 2 1 4 5 6  ), (3 1 2 4 5 6  ), (2 3 1 4 5 6  ), (1 3 2 4 5 6  ), (2 1 3 4 5 6  ), (1 2 3 4 5 6  )
\end{enumerate}
Invariant group of Orbit $2$:
\begin{enumerate}
\item Group Size is $48$.
\item Elements are : (4 3 2 1 6 5  ), (4 3 1 2 6 5  ), (4 2 3 1 6 5  ), (4 1 3 2 6 5  ), (4 2 1 3 6 5  ), (4 1 2 3 6 5  ), (3 4 2 1 6 5  ), (3 4 1 2 6 5  ), (2 4 3 1 6 5  ), (1 4 3 2 6 5  ), (2 4 1 3 6 5  ), (1 4 2 3 6 5  ), (3 2 4 1 6 5  ), (3 1 4 2 6 5  ), (2 3 4 1 6 5  ), (1 3 4 2 6 5  ), (2 1 4 3 6 5  ), (1 2 4 3 6 5  ), (3 2 1 4 6 5  ), (3 1 2 4 6 5  ), (2 3 1 4 6 5  ), (1 3 2 4 6 5  ), (2 1 3 4 6 5  ), (1 2 3 4 6 5  ), (4 3 2 1 5 6  ), (4 3 1 2 5 6  ), (4 2 3 1 5 6  ), (4 1 3 2 5 6  ), (4 2 1 3 5 6  ), (4 1 2 3 5 6  ), (3 4 2 1 5 6  ), (3 4 1 2 5 6  ), (2 4 3 1 5 6  ), (1 4 3 2 5 6  ), (2 4 1 3 5 6  ), (1 4 2 3 5 6  ), (3 2 4 1 5 6  ), (3 1 4 2 5 6  ), (2 3 4 1 5 6  ), (1 3 4 2 5 6  ), (2 1 4 3 5 6  ), (1 2 4 3 5 6  ), (3 2 1 4 5 6  ), (3 1 2 4 5 6  ), (2 3 1 4 5 6  ), (1 3 2 4 5 6  ), (2 1 3 4 5 6  ), (1 2 3 4 5 6  )
\end{enumerate}
Invariant group of Orbit $3$:
\begin{enumerate}
\item Group Size is $72$.
\item Elements are : (6 5 4 3 2 1  ), (6 5 4 3 1 2  ), (6 5 4 2 3 1  ), (6 5 4 1 3 2  ), (6 5 4 2 1 3  ), (6 5 4 1 2 3  ), (6 4 5 3 2 1  ), (6 4 5 3 1 2  ), (6 4 5 2 3 1  ), (6 4 5 1 3 2  ), (6 4 5 2 1 3  ), (6 4 5 1 2 3  ), (5 6 4 3 2 1  ), (5 6 4 3 1 2  ), (5 6 4 2 3 1  ), (5 6 4 1 3 2  ), (5 6 4 2 1 3  ), (5 6 4 1 2 3  ), (4 6 5 3 2 1  ), (4 6 5 3 1 2  ), (4 6 5 2 3 1  ), (4 6 5 1 3 2  ), (4 6 5 2 1 3  ), (4 6 5 1 2 3  ), (5 4 6 3 2 1  ), (5 4 6 3 1 2  ), (5 4 6 2 3 1  ), (5 4 6 1 3 2  ), (5 4 6 2 1 3  ), (5 4 6 1 2 3  ), (4 5 6 3 2 1  ), (4 5 6 3 1 2  ), (4 5 6 2 3 1  ), (4 5 6 1 3 2  ), (4 5 6 2 1 3  ), (4 5 6 1 2 3  ), (3 2 1 6 5 4  ), (3 1 2 6 5 4  ), (2 3 1 6 5 4  ), (1 3 2 6 5 4  ), (2 1 3 6 5 4  ), (1 2 3 6 5 4  ), (3 2 1 6 4 5  ), (3 1 2 6 4 5  ), (2 3 1 6 4 5  ), (1 3 2 6 4 5  ), (2 1 3 6 4 5  ), (1 2 3 6 4 5  ), (3 2 1 5 6 4  ), (3 1 2 5 6 4  ), (2 3 1 5 6 4  ), (1 3 2 5 6 4  ), (2 1 3 5 6 4  ), (1 2 3 5 6 4  ), (3 2 1 4 6 5  ), (3 1 2 4 6 5  ), (2 3 1 4 6 5  ), (1 3 2 4 6 5  ), (2 1 3 4 6 5  ), (1 2 3 4 6 5  ), (3 2 1 5 4 6  ), (3 1 2 5 4 6  ), (2 3 1 5 4 6  ), (1 3 2 5 4 6  ), (2 1 3 5 4 6  ), (1 2 3 5 4 6  ), (3 2 1 4 5 6  ), (3 1 2 4 5 6  ), (2 3 1 4 5 6  ), (1 3 2 4 5 6  ), (2 1 3 4 5 6  ), (1 2 3 4 5 6  )
\end{enumerate}
Invariant group of Orbit $4$:
\begin{enumerate}
\item Group Size is $72$.
\item Elements are : (6 4 3 2 5 1  ), (6 4 2 3 5 1  ), (6 3 4 2 5 1  ), (6 2 4 3 5 1  ), (6 3 2 4 5 1  ), (6 2 3 4 5 1  ), (6 4 3 2 1 5  ), (6 4 2 3 1 5  ), (6 3 4 2 1 5  ), (6 2 4 3 1 5  ), (6 3 2 4 1 5  ), (6 2 3 4 1 5  ), (4 6 5 1 3 2  ), (4 6 5 1 2 3  ), (3 6 5 1 4 2  ), (2 6 5 1 4 3  ), (3 6 5 1 2 4  ), (2 6 5 1 3 4  ), (4 6 1 5 3 2  ), (4 6 1 5 2 3  ), (3 6 1 5 4 2  ), (2 6 1 5 4 3  ), (3 6 1 5 2 4  ), (2 6 1 5 3 4  ), (4 5 6 1 3 2  ), (4 5 6 1 2 3  ), (3 5 6 1 4 2  ), (2 5 6 1 4 3  ), (3 5 6 1 2 4  ), (2 5 6 1 3 4  ), (4 1 6 5 3 2  ), (4 1 6 5 2 3  ), (3 1 6 5 4 2  ), (2 1 6 5 4 3  ), (3 1 6 5 2 4  ), (2 1 6 5 3 4  ), (4 5 1 6 3 2  ), (4 5 1 6 2 3  ), (3 5 1 6 4 2  ), (2 5 1 6 4 3  ), (3 5 1 6 2 4  ), (2 5 1 6 3 4  ), (4 1 5 6 3 2  ), (4 1 5 6 2 3  ), (3 1 5 6 4 2  ), (2 1 5 6 4 3  ), (3 1 5 6 2 4  ), (2 1 5 6 3 4  ), (5 4 3 2 6 1  ), (5 4 2 3 6 1  ), (5 3 4 2 6 1  ), (5 2 4 3 6 1  ), (5 3 2 4 6 1  ), (5 2 3 4 6 1  ), (1 4 3 2 6 5  ), (1 4 2 3 6 5  ), (1 3 4 2 6 5  ), (1 2 4 3 6 5  ), (1 3 2 4 6 5  ), (1 2 3 4 6 5  ), (5 4 3 2 1 6  ), (5 4 2 3 1 6  ), (5 3 4 2 1 6  ), (5 2 4 3 1 6  ), (5 3 2 4 1 6  ), (5 2 3 4 1 6  ), (1 4 3 2 5 6  ), (1 4 2 3 5 6  ), (1 3 4 2 5 6  ), (1 2 4 3 5 6  ), (1 3 2 4 5 6  ), (1 2 3 4 5 6  )
\end{enumerate}
Invariant group of Orbit $5$:
\begin{enumerate}
\item Group Size is $12$.
\item Elements are : (2 1 5 4 3 6  ), (1 2 5 4 3 6  ), (2 1 5 3 4 6  ), (1 2 5 3 4 6  ), (2 1 4 5 3 6  ), (1 2 4 5 3 6  ), (2 1 3 5 4 6  ), (1 2 3 5 4 6  ), (2 1 4 3 5 6  ), (1 2 4 3 5 6  ), (2 1 3 4 5 6  ), (1 2 3 4 5 6  )
\end{enumerate}
Invariant group of Orbit $6$:
\begin{enumerate}
\item Group Size is $48$.
\item Elements are : (6 5 4 3 2 1  ), (6 5 4 2 3 1  ), (6 5 3 4 2 1  ), (6 5 2 4 3 1  ), (6 5 3 2 4 1  ), (6 5 2 3 4 1  ), (6 4 5 3 2 1  ), (6 4 5 2 3 1  ), (6 3 5 4 2 1  ), (6 2 5 4 3 1  ), (6 3 5 2 4 1  ), (6 2 5 3 4 1  ), (6 4 3 5 2 1  ), (6 4 2 5 3 1  ), (6 3 4 5 2 1  ), (6 2 4 5 3 1  ), (6 3 2 5 4 1  ), (6 2 3 5 4 1  ), (6 4 3 2 5 1  ), (6 4 2 3 5 1  ), (6 3 4 2 5 1  ), (6 2 4 3 5 1  ), (6 3 2 4 5 1  ), (6 2 3 4 5 1  ), (1 5 4 3 2 6  ), (1 5 4 2 3 6  ), (1 5 3 4 2 6  ), (1 5 2 4 3 6  ), (1 5 3 2 4 6  ), (1 5 2 3 4 6  ), (1 4 5 3 2 6  ), (1 4 5 2 3 6  ), (1 3 5 4 2 6  ), (1 2 5 4 3 6  ), (1 3 5 2 4 6  ), (1 2 5 3 4 6  ), (1 4 3 5 2 6  ), (1 4 2 5 3 6  ), (1 3 4 5 2 6  ), (1 2 4 5 3 6  ), (1 3 2 5 4 6  ), (1 2 3 5 4 6  ), (1 4 3 2 5 6  ), (1 4 2 3 5 6  ), (1 3 4 2 5 6  ), (1 2 4 3 5 6  ), (1 3 2 4 5 6  ), (1 2 3 4 5 6  )
\end{enumerate}
Invariant group of Orbit $7$:
\begin{enumerate}
\item Group Size is $8$.
\item Elements are : (2 1 4 3 6 5  ), (1 2 4 3 6 5  ), (2 1 3 4 6 5  ), (1 2 3 4 6 5  ), (2 1 4 3 5 6  ), (1 2 4 3 5 6  ), (2 1 3 4 5 6  ), (1 2 3 4 5 6  )
\end{enumerate}
Invariant group of Orbit $8$:
\begin{enumerate}
\item Group Size is $8$.
\item Elements are : (3 5 1 6 2 4  ), (2 5 1 6 3 4  ), (3 1 5 6 2 4  ), (2 1 5 6 3 4  ), (5 3 2 4 1 6  ), (5 2 3 4 1 6  ), (1 3 2 4 5 6  ), (1 2 3 4 5 6  )
\end{enumerate}
\section{The intrisic informations concerning the graph of facet rays of MET6.ine}
The graph considered is made of $60$ elements.\\
The group acting on MET6.ine is ReprS6\\
Under the group action we have in fact only $1$ orbits to consider.\\
The diameter of the graph is $2$.\\
The list of orbits is with adjacency and Size info:
\begin{center}
\scriptsize
\begin{tabular}{cccccccccccccccc|c|c}
F1:&-1&0&0&0&1&0&0&0&1&0&0&0&0&0&0&45&60\\
\end{tabular}
\end{center}
The representation matrix is:
\begin{center}
\scriptsize
\begin{tabular}{|c|c|c|c|}
\hline
&F1&Adj.&Size\\
\hline
F1& 45&45&60\\
\hline
Size&60&&60\\
\hline
\end{tabular}
\end{center}
No complemented local graph of size inferior to $20$.
\subsection{scalar product between elements of orbits}
\noindent This subsection made sense only if the representation is orthogonal which is the usual case
Considering Orbit 1
\begin{enumerate}
\item Orbit Size is $60$.
\item Norm of elements is $3$.
\item Possible scalar product : $-1$, $0$, $1$
\end{enumerate}
\subsection{invariant group of Orbits}
\noindent The invariant group of $x$ is the set of $g$ such that $g(x)=x$ denoted by $G_x$.\\
If $x$ and $y$ are in the same orbit then the groups $G_x$ and  $G_y$ are conjuguate.\\
For any element $x$ we denote $O_x$ its Orbit, we have $|O_x|\times |G_x|=|G|$
We print only nontrivial groups
Invariant group of Orbit $1$:
\begin{enumerate}
\item Group Size is $12$.
\item Elements are : (2 1 5 4 3 6  ), (1 2 5 4 3 6  ), (2 1 5 3 4 6  ), (1 2 5 3 4 6  ), (2 1 4 5 3 6  ), (1 2 4 5 3 6  ), (2 1 3 5 4 6  ), (1 2 3 5 4 6  ), (2 1 4 3 5 6  ), (1 2 4 3 5 6  ), (2 1 3 4 5 6  ), (1 2 3 4 5 6  )
\end{enumerate}
\section{The two incidence matrix}
Incidence between Orbit of Extreme rays and Orbits of Facets
\begin{equation*}
\begin{array}{|c|c|c|}
\hline
&F1&Inc\\
\hline
E1&50&50\\
E2&44&44\\
E3&42&42\\
E4&18&18\\
E5&23&23\\
E6&16&16\\
E7&22&22\\
E8&18&18\\
\hline
\end{array}
\end{equation*}
Incidence between Orbits of Facets and Orbits of Extreme Rays
\begin{equation*}
\begin{array}{|c|cccccccc|c|}
\hline
&E1&E2&E3&E4&E5&E6&E7&E8&Inc\\
\hline
F1&5&11&7&3&23&4&33&27&113\\
\hline
\end{array}
\end{equation*}
\end{document}
