\documentclass[12pt]{article}
\usepackage{graphicx,amssymb,amsmath,vmargin,multicol}
\usepackage[english]{babel}
\title{The Cone MET5}
\setpapersize{custom}{21cm}{29.7cm}
\setmarginsrb{1cm}{1cm}{1cm}{2cm}{0pt}{0pt}{0pt}{0pt}
\begin{document}
\maketitle
\section{The intrisic informations concerning the graph of extreme rays of MET5.ext}
The graph considered is made of $25$ elements.\\
The group acting on MET5.ext is ReprS5\\
Under the group action we have in fact only $3$ orbits to consider.\\
The diameter of the graph is $2$.\\
The list of orbits is with adjacency and Size info:
\begin{center}
\scriptsize
\begin{tabular}{ccccccccccc|c|c}
E1:&0&0&1&1&0&1&1&1&1&0&20&10\\
E2:&0&0&0&1&0&0&1&0&1&1&20&5\\
E3:&1&1&1&2&2&2&1&2&1&1&9&10\\
\end{tabular}
\end{center}
The representation matrix is:
\begin{center}
\scriptsize
\begin{tabular}{|c|ccc|c|c|}
\hline
&E1&E2&E3&Adj.&Size\\
\hline
E1& 9& 5& 6&20&10\\
E2& 10& 4& 6&20&5\\
E3& 6& 3& 0&9&10\\
\hline
Size&10&5&10&&25\\
\hline
\end{tabular}
\end{center}
\subsection{The local graphs}
The complemented local graph are:\\
We consider graph with cardinal inferior or equal to $20$\\
The complemented local graph for orbit $1$ is:
\begin{equation*}
\begin{array}{rrcl}
1&2&:&\,\,9\,\,10\\
2&3&:&\,\,5\,\,7\,\,9\\
3&11&:&\,\,5\,\,6\,\,7\,\,8\,\,9\,\,10\,\,11\,\,12\,\,15\,\,17\,\,19\\
4&2&:&\,\,9\,\,10\\
5&11&:&\,\,2\,\,3\,\,6\,\,7\,\,9\,\,10\,\,11\,\,14\,\,17\,\,18\,\,20\\
6&11&:&\,\,3\,\,5\,\,7\,\,8\,\,9\,\,10\,\,13\,\,14\,\,16\,\,18\,\,19\\
7&11&:&\,\,2\,\,3\,\,5\,\,6\,\,9\,\,10\,\,12\,\,13\,\,15\,\,16\,\,20\\
8&3&:&\,\,3\,\,6\,\,10\\
9&11&:&\,\,1\,\,2\,\,3\,\,4\,\,5\,\,6\,\,7\,\,10\,\,13\,\,17\,\,19\\
10&11&:&\,\,1\,\,3\,\,4\,\,5\,\,6\,\,7\,\,8\,\,9\,\,11\,\,16\,\,20\\
11&3&:&\,\,3\,\,5\,\,10\\
12&2&:&\,\,3\,\,7\\
13&3&:&\,\,6\,\,7\,\,9\\
14&2&:&\,\,5\,\,6\\
15&2&:&\,\,3\,\,7\\
16&3&:&\,\,6\,\,7\,\,10\\
17&3&:&\,\,3\,\,5\,\,9\\
18&2&:&\,\,5\,\,6\\
19&3&:&\,\,3\,\,6\,\,9\\
20&3&:&\,\,5\,\,7\,\,10\\
\end{array}
\end{equation*}
The complemented local graph for orbit $2$ is:
\begin{equation*}
\begin{array}{rrcl}
1&2&:&\,\,9\,\,10\\
2&2&:&\,\,9\,\,10\\
3&11&:&\,\,4\,\,5\,\,6\,\,7\,\,9\,\,10\,\,11\,\,12\,\,13\,\,18\,\,19\\
4&11&:&\,\,3\,\,6\,\,7\,\,8\,\,9\,\,10\,\,11\,\,12\,\,15\,\,17\,\,20\\
5&3&:&\,\,3\,\,7\,\,10\\
6&11&:&\,\,3\,\,4\,\,7\,\,8\,\,9\,\,10\,\,13\,\,14\,\,16\,\,19\,\,20\\
7&11&:&\,\,3\,\,4\,\,5\,\,6\,\,9\,\,10\,\,14\,\,15\,\,16\,\,17\,\,18\\
8&3&:&\,\,4\,\,6\,\,9\\
9&11&:&\,\,1\,\,2\,\,3\,\,4\,\,6\,\,7\,\,8\,\,10\,\,15\,\,18\,\,19\\
10&11&:&\,\,1\,\,2\,\,3\,\,4\,\,5\,\,6\,\,7\,\,9\,\,13\,\,17\,\,20\\
11&2&:&\,\,3\,\,4\\
12&2&:&\,\,3\,\,4\\
13&3&:&\,\,3\,\,6\,\,10\\
14&2&:&\,\,6\,\,7\\
15&3&:&\,\,4\,\,7\,\,9\\
16&2&:&\,\,6\,\,7\\
17&3&:&\,\,4\,\,7\,\,10\\
18&3&:&\,\,3\,\,7\,\,9\\
19&3&:&\,\,3\,\,6\,\,9\\
20&3&:&\,\,4\,\,6\,\,10\\
\end{array}
\end{equation*}
The complemented local graph for orbit $3$ is:
\begin{equation*}
\begin{array}{rrcl}
1&0&:&\\
2&0&:&\\
3&0&:&\\
4&0&:&\\
5&0&:&\\
6&0&:&\\
7&0&:&\\
8&0&:&\\
9&0&:&\\
\end{array}
\end{equation*}
\subsection{scalar product between elements of orbits}
\noindent This subsection made sense only if the representation is orthogonal which is the usual case
Considering Orbit 1
\begin{enumerate}
\item Orbit Size is $10$.
\item Norm of elements is $6$.
\item Possible scalar product : $3$, $4$
\end{enumerate}
Considering Orbit 2
\begin{enumerate}
\item Orbit Size is $5$.
\item Norm of elements is $4$.
\item Possible scalar product : $1$
\end{enumerate}
Considering Orbit 3
\begin{enumerate}
\item Orbit Size is $10$.
\item Norm of elements is $22$.
\item Possible scalar product : $19$, $20$
\end{enumerate}
\subsection{invariant group of Orbits}
\noindent The invariant group of $x$ is the set of $g$ such that $g(x)=x$ denoted by $G_x$.\\
If $x$ and $y$ are in the same orbit then the groups $G_x$ and  $G_y$ are conjuguate.\\
For any element $x$ we denote $O_x$ its Orbit, we have $|O_x|\times |G_x|=|G|$
We print only nontrivial groups
Invariant group of Orbit $1$:
\begin{enumerate}
\item Group Size is $12$.
\item Elements are : (3 2 1 5 4  ), (3 1 2 5 4  ), (2 3 1 5 4  ), (1 3 2 5 4  ), (2 1 3 5 4  ), (1 2 3 5 4  ), (3 2 1 4 5  ), (3 1 2 4 5  ), (2 3 1 4 5  ), (1 3 2 4 5  ), (2 1 3 4 5  ), (1 2 3 4 5  )
\end{enumerate}
Invariant group of Orbit $2$:
\begin{enumerate}
\item Group Size is $24$.
\item Elements are : (4 3 2 1 5  ), (4 3 1 2 5  ), (4 2 3 1 5  ), (4 1 3 2 5  ), (4 2 1 3 5  ), (4 1 2 3 5  ), (3 4 2 1 5  ), (3 4 1 2 5  ), (2 4 3 1 5  ), (1 4 3 2 5  ), (2 4 1 3 5  ), (1 4 2 3 5  ), (3 2 4 1 5  ), (3 1 4 2 5  ), (2 3 4 1 5  ), (1 3 4 2 5  ), (2 1 4 3 5  ), (1 2 4 3 5  ), (3 2 1 4 5  ), (3 1 2 4 5  ), (2 3 1 4 5  ), (1 3 2 4 5  ), (2 1 3 4 5  ), (1 2 3 4 5  )
\end{enumerate}
Invariant group of Orbit $3$:
\begin{enumerate}
\item Group Size is $12$.
\item Elements are : (5 4 3 2 1  ), (5 4 2 3 1  ), (5 3 4 2 1  ), (5 2 4 3 1  ), (5 3 2 4 1  ), (5 2 3 4 1  ), (1 4 3 2 5  ), (1 4 2 3 5  ), (1 3 4 2 5  ), (1 2 4 3 5  ), (1 3 2 4 5  ), (1 2 3 4 5  )
\end{enumerate}
\section{The intrisic informations concerning the graph of facet rays of MET5.ine}
The graph considered is made of $30$ elements.\\
The group acting on MET5.ine is ReprS5\\
Under the group action we have in fact only $1$ orbits to consider.\\
The diameter of the graph is $2$.\\
The list of orbits is with adjacency and Size info:
\begin{center}
\scriptsize
\begin{tabular}{ccccccccccc|c|c}
F1:&-1&0&0&1&0&0&1&0&0&0&19&30\\
\end{tabular}
\end{center}
The representation matrix is:
\begin{center}
\scriptsize
\begin{tabular}{|c|c|c|c|}
\hline
&F1&Adj.&Size\\
\hline
F1& 19&19&30\\
\hline
Size&30&&30\\
\hline
\end{tabular}
\end{center}
\subsection{The local graphs}
The complemented local graph are:\\
We consider graph with cardinal inferior or equal to $20$\\
The complemented local graph for orbit $1$ is:
\begin{equation*}
\begin{array}{rrcl}
1&4&:&\,\,5\,\,8\,\,13\,\,17\\
2&4&:&\,\,3\,\,4\,\,6\,\,7\\
3&6&:&\,\,2\,\,10\,\,14\,\,17\,\,18\,\,19\\
4&7&:&\,\,2\,\,5\,\,9\,\,15\,\,17\,\,18\,\,19\\
5&6&:&\,\,1\,\,4\,\,11\,\,12\,\,16\,\,19\\
6&6&:&\,\,2\,\,10\,\,13\,\,14\,\,15\,\,18\\
7&7&:&\,\,2\,\,8\,\,9\,\,13\,\,14\,\,15\,\,19\\
8&6&:&\,\,1\,\,7\,\,11\,\,12\,\,15\,\,16\\
9&10&:&\,\,4\,\,7\,\,10\,\,11\,\,12\,\,13\,\,14\,\,16\,\,17\,\,18\\
10&8&:&\,\,3\,\,6\,\,9\,\,11\,\,12\,\,15\,\,16\,\,19\\
11&8&:&\,\,5\,\,8\,\,9\,\,10\,\,14\,\,15\,\,18\,\,19\\
12&5&:&\,\,5\,\,8\,\,9\,\,10\,\,13\\
13&7&:&\,\,1\,\,6\,\,7\,\,9\,\,12\,\,15\,\,19\\
14&5&:&\,\,3\,\,6\,\,7\,\,9\,\,11\\
15&8&:&\,\,4\,\,6\,\,7\,\,8\,\,10\,\,11\,\,13\,\,17\\
16&5&:&\,\,5\,\,8\,\,9\,\,10\,\,17\\
17&7&:&\,\,1\,\,3\,\,4\,\,9\,\,15\,\,16\,\,19\\
18&5&:&\,\,3\,\,4\,\,6\,\,9\,\,11\\
19&8&:&\,\,3\,\,4\,\,5\,\,7\,\,10\,\,11\,\,13\,\,17\\
\end{array}
\end{equation*}
\subsection{scalar product between elements of orbits}
\noindent This subsection made sense only if the representation is orthogonal which is the usual case
Considering Orbit 1
\begin{enumerate}
\item Orbit Size is $30$.
\item Norm of elements is $3$.
\item Possible scalar product : $-1$, $0$, $1$
\end{enumerate}
\subsection{invariant group of Orbits}
\noindent The invariant group of $x$ is the set of $g$ such that $g(x)=x$ denoted by $G_x$.\\
If $x$ and $y$ are in the same orbit then the groups $G_x$ and  $G_y$ are conjuguate.\\
For any element $x$ we denote $O_x$ its Orbit, we have $|O_x|\times |G_x|=|G|$
We print only nontrivial groups
Invariant group of Orbit $1$:
\begin{enumerate}
\item Group Size is $4$.
\item Elements are : (2 1 4 3 5  ), (1 2 4 3 5  ), (2 1 3 4 5  ), (1 2 3 4 5  )
\end{enumerate}
\section{The two incidence matrix}
Incidence between Orbit of Extreme rays and Orbits of Facets
\begin{equation*}
\begin{array}{|c|c|c|}
\hline
&F1&Inc\\
\hline
E1&21&21\\
E2&24&24\\
E3&9&9\\
\hline
\end{array}
\end{equation*}
Incidence between Orbits of Facets and Orbits of Extreme Rays
\begin{equation*}
\begin{array}{|c|ccc|c|}
\hline
&E1&E2&E3&Inc\\
\hline
F1&7&4&3&14\\
\hline
\end{array}
\end{equation*}
\end{document}
