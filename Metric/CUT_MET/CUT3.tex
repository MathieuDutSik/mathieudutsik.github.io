\documentclass[12pt]{article}
\usepackage{graphicx,amssymb,amsmath,vmargin,multicol}
\usepackage[english]{babel}
\title{The Cone CUT3}
\setpapersize{custom}{21cm}{29.7cm}
\setmarginsrb{1cm}{1cm}{1cm}{2cm}{0pt}{0pt}{0pt}{0pt}
\begin{document}
\maketitle
\section{The intrisic informations concerning the graph of extreme rays of CUT3.ext}
The graph considered is made of $3$ elements.\\
The group acting on CUT3.ext is ReprS3\\
Under the group action we have in fact only $1$ orbits to consider.\\
The diameter of the graph is $1$.\\
The list of orbits is with adjacency and Size info:
\begin{center}
\scriptsize
\begin{tabular}{cccc|c|c}
E1:&0&1&1&2&3\\
\end{tabular}
\end{center}
The representation matrix is:
\begin{center}
\scriptsize
\begin{tabular}{|c|c|c|c|}
\hline
&E1&Adj.&Size\\
\hline
E1& 2&2&3\\
\hline
Size&3&&3\\
\hline
\end{tabular}
\end{center}
\subsection{The local graphs}
The complemented local graph are:\\
We consider graph with cardinal inferior or equal to $20$\\
The complemented local graph for orbit $1$ is:
\begin{equation*}
\begin{array}{rrcl}
1&0&:&\\
2&0&:&\\
\end{array}
\end{equation*}
\subsection{scalar product between elements of orbits}
\noindent This subsection made sense only if the representation is orthogonal which is the usual case
Considering Orbit 1
\begin{enumerate}
\item Orbit Size is $3$.
\item Norm of elements is $2$.
\item Possible scalar product : $1$
\end{enumerate}
\subsection{invariant group of Orbits}
\noindent The invariant group of $x$ is the set of $g$ such that $g(x)=x$ denoted by $G_x$.\\
If $x$ and $y$ are in the same orbit then the groups $G_x$ and  $G_y$ are conjuguate.\\
For any element $x$ we denote $O_x$ its Orbit, we have $|O_x|\times |G_x|=|G|$
We print only nontrivial groups
Invariant group of Orbit $1$:
\begin{enumerate}
\item Group Size is $2$.
\item Elements are : (2 1 3  ), (1 2 3  )
\end{enumerate}
\section{The intrisic informations concerning the graph of facet rays of CUT3.ine}
The graph considered is made of $3$ elements.\\
The group acting on CUT3.ine is ReprS3\\
Under the group action we have in fact only $1$ orbits to consider.\\
The diameter of the graph is $1$.\\
The list of orbits is with adjacency and Size info:
\begin{center}
\scriptsize
\begin{tabular}{cccc|c|c}
F1:&-1&1&1&2&3\\
\end{tabular}
\end{center}
The representation matrix is:
\begin{center}
\scriptsize
\begin{tabular}{|c|c|c|c|}
\hline
&F1&Adj.&Size\\
\hline
F1& 2&2&3\\
\hline
Size&3&&3\\
\hline
\end{tabular}
\end{center}
\subsection{The local graphs}
The complemented local graph are:\\
We consider graph with cardinal inferior or equal to $20$\\
The complemented local graph for orbit $1$ is:
\begin{equation*}
\begin{array}{rrcl}
1&0&:&\\
2&0&:&\\
\end{array}
\end{equation*}
\subsection{scalar product between elements of orbits}
\noindent This subsection made sense only if the representation is orthogonal which is the usual case
Considering Orbit 1
\begin{enumerate}
\item Orbit Size is $3$.
\item Norm of elements is $3$.
\item Possible scalar product : $-1$
\end{enumerate}
\subsection{invariant group of Orbits}
\noindent The invariant group of $x$ is the set of $g$ such that $g(x)=x$ denoted by $G_x$.\\
If $x$ and $y$ are in the same orbit then the groups $G_x$ and  $G_y$ are conjuguate.\\
For any element $x$ we denote $O_x$ its Orbit, we have $|O_x|\times |G_x|=|G|$
We print only nontrivial groups
Invariant group of Orbit $1$:
\begin{enumerate}
\item Group Size is $2$.
\item Elements are : (2 1 3  ), (1 2 3  )
\end{enumerate}
\section{The two incidence matrix}
Incidence between Orbit of Extreme rays and Orbits of Facets
\begin{equation*}
\begin{array}{|c|c|c|}
\hline
&F1&Inc\\
\hline
E1&2&2\\
\hline
\end{array}
\end{equation*}
Incidence between Orbits of Facets and Orbits of Extreme Rays
\begin{equation*}
\begin{array}{|c|c|c|}
\hline
&E1&Inc\\
\hline
F1&2&2\\
\hline
\end{array}
\end{equation*}
\end{document}
