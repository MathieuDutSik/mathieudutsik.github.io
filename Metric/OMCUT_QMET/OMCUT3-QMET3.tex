\documentclass[12pt]{article}
\usepackage{graphicx,amssymb,amsmath,vmargin,multicol}
\usepackage[english]{babel}
\title{The Cone OMCUT3-QMET3}
\setpapersize{custom}{21cm}{29.7cm}
\setmarginsrb{1cm}{1cm}{1cm}{2cm}{0pt}{0pt}{0pt}{0pt}
\begin{document}
\maketitle
\section{The intrisic informations concerning the graph of extreme rays of OMCUT3-QMET3.ext}
The graph considered is made of $12$ elements.\\
The group acting on OMCUT3-QMET3.ext is ReprS3revers\\
Under the group action we have in fact only $2$ orbits to consider.\\
The diameter of the graph is $2$.\\
The list of orbits is with adjacency and Size info:
\begin{center}
\scriptsize
\begin{tabular}{ccccccc|c|c}
E1:&0&0&0&0&1&1&9&6\\
E2:&0&0&1&0&1&1&6&6\\
\end{tabular}
\end{center}
The representation matrix is:
\begin{center}
\scriptsize
\begin{tabular}{|c|cc|c|c|}
\hline
&E1&E2&Adj.&Size\\
\hline
E1& 5& 4&9&6\\
E2& 4& 2&6&6\\
\hline
Size&6&6&&12\\
\hline
\end{tabular}
\end{center}
\subsection{The local graphs}
The complemented local graph are:\\
We consider graph with cardinal inferior or equal to $20$\\
The complemented local graph for orbit $1$ is:
\begin{equation*}
\begin{array}{rrcl}
1&1&:&\,\,5\\
2&4&:&\,\,4\,\,5\,\,6\,\,7\\
3&4&:&\,\,5\,\,6\,\,8\,\,9\\
4&2&:&\,\,2\,\,6\\
5&4&:&\,\,1\,\,2\,\,3\,\,8\\
6&4&:&\,\,2\,\,3\,\,4\,\,9\\
7&1&:&\,\,2\\
8&2&:&\,\,3\,\,5\\
9&2&:&\,\,3\,\,6\\
\end{array}
\end{equation*}
The complemented local graph for orbit $2$ is:
\begin{equation*}
\begin{array}{rrcl}
1&1&:&\,\,2\\
2&2&:&\,\,1\,\,5\\
3&1&:&\,\,4\\
4&2&:&\,\,3\,\,6\\
5&1&:&\,\,2\\
6&1&:&\,\,4\\
\end{array}
\end{equation*}
\subsection{scalar product between elements of orbits}
\noindent This subsection made sense only if the representation is orthogonal which is the usual case
Considering Orbit 1
\begin{enumerate}
\item Orbit Size is $6$.
\item Norm of elements is $2$.
\item Possible scalar product : $0$, $1$
\end{enumerate}
Considering Orbit 2
\begin{enumerate}
\item Orbit Size is $6$.
\item Norm of elements is $3$.
\item Possible scalar product : $0$, $1$, $2$
\end{enumerate}
\subsection{invariant group of Orbits}
\noindent The invariant group of $x$ is the set of $g$ such that g(x)=x denoted b $G_x$.\\
If $x$ and $y$ are in the same orbit then the groups $G_x$ and  $G_x$ are conjuguate.\\
Invariant group of Orbit 1
\begin{enumerate}
\item Group Size is $2$.
\item Elements are : (6 5 4 3 2 1  reversal), (4 3 6 5 1 2  reversal), 
\end{enumerate}
Invariant group of Orbit 2
\begin{enumerate}
\item Group Size is $2$.
\item Elements are : (6 5 4 3 2 1  reversal), (2 1 5 6 3 4  reversal), 
\end{enumerate}
\section{The intrisic informations concerning the graph of facet rays of OMCUT3-QMET3.ine}
The graph considered is made of $12$ elements.\\
The group acting on OMCUT3-QMET3.ine is ReprS3revers\\
Under the group action we have in fact only $2$ orbits to consider.\\
The diameter of the graph is $2$.\\
The list of orbits is with adjacency and Size info:
\begin{center}
\scriptsize
\begin{tabular}{ccccccc|c|c}
F1:&-1&1&0&0&0&1&8&6\\
F2:&0&0&0&0&0&1&7&6\\
\end{tabular}
\end{center}
The representation matrix is:
\begin{center}
\scriptsize
\begin{tabular}{|c|cc|c|c|}
\hline
&F1&F2&Adj.&Size\\
\hline
F1& 3& 5&8&6\\
F2& 5& 2&7&6\\
\hline
Size&6&6&&12\\
\hline
\end{tabular}
\end{center}
\subsection{The local graphs}
The complemented local graph are:\\
We consider graph with cardinal inferior or equal to $20$\\
The complemented local graph for orbit $1$ is:
\begin{equation*}
\begin{array}{rrcl}
1&3&:&\,\,2\,\,3\,\,5\\
2&2&:&\,\,1\,\,6\\
3&2&:&\,\,1\,\,7\\
4&3&:&\,\,5\,\,7\,\,8\\
5&3&:&\,\,1\,\,4\,\,8\\
6&3&:&\,\,2\,\,7\,\,8\\
7&3&:&\,\,3\,\,4\,\,6\\
8&3&:&\,\,4\,\,5\,\,6\\
\end{array}
\end{equation*}
The complemented local graph for orbit $2$ is:
\begin{equation*}
\begin{array}{rrcl}
1&2&:&\,\,2\,\,6\\
2&2&:&\,\,1\,\,4\\
3&2&:&\,\,4\,\,5\\
4&2&:&\,\,2\,\,3\\
5&2&:&\,\,3\,\,7\\
6&2&:&\,\,1\,\,7\\
7&2&:&\,\,5\,\,6\\
\end{array}
\end{equation*}
\subsection{scalar product between elements of orbits}
\noindent This subsection made sense only if the representation is orthogonal which is the usual case
Considering Orbit 1
\begin{enumerate}
\item Orbit Size is $6$.
\item Norm of elements is $3$.
\item Possible scalar product : $-2$, $0$, $1$
\end{enumerate}
Considering Orbit 2
\begin{enumerate}
\item Orbit Size is $6$.
\item Norm of elements is $1$.
\item Possible scalar product : $0$
\end{enumerate}
\subsection{invariant group of Orbits}
\noindent The invariant group of $x$ is the set of $g$ such that g(x)=x denoted b $G_x$.\\
If $x$ and $y$ are in the same orbit then the groups $G_x$ and  $G_x$ are conjuguate.\\
Invariant group of Orbit 1
\begin{enumerate}
\item Group Size is $2$.
\item Elements are : (6 5 4 3 2 1  reversal), (6 5 4 3 2 1 ), 
\end{enumerate}
Invariant group of Orbit 2
\begin{enumerate}
\item Group Size is $2$.
\item Elements are : (6 5 4 3 2 1  reversal), (3 4 1 2 6 5  reversal), 
\end{enumerate}
\section{The two incidence matrix}
Incidence between Orbit of Extreme rays and Orbits of Facets
\begin{equation*}
\begin{array}{|c|cc|c|}
\hline
&F1&F2&Inc\\
\hline
E1&4&4&8\\
E2&3&3&6\\
\hline
\end{array}
\end{equation*}
Incidence between Orbits of Facets and Orbits of Extreme Rays
\begin{equation*}
\begin{array}{|c|cc|c|}
\hline
&E1&E2&Inc\\
\hline
F1&4&3&7\\
F2&4&3&7\\
\hline
\end{array}
\end{equation*}
\end{document}
