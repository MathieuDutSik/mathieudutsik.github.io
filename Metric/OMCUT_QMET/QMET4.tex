\documentclass[12pt]{article}
\usepackage{graphicx,amssymb,amsmath,vmargin,multicol}
\usepackage[english]{babel}
\title{The Cone QMET4}
\setpapersize{custom}{21cm}{29.7cm}
\setmarginsrb{1cm}{1cm}{1cm}{2cm}{0pt}{0pt}{0pt}{0pt}
\begin{document}
\maketitle
\section{The intrisic informations concerning the graph of extreme rays of QMET4.ext}
The graph considered is made of $164$ elements.\\
The group acting on QMET4.ext is ReprS4revers\\
Under the group action we have in fact only $10$ orbits to consider.\\
The diameter of the graph is $3$.\\
The list of orbits is with adjacency and Size info:
\begin{center}
\scriptsize
\begin{tabular}{ccccccccccccc|c|c}
E1:&0&0&0&0&0&0&1&1&0&1&1&0&97&6\\
E2:&0&0&0&0&0&0&0&0&0&1&1&1&91&8\\
E3:&0&0&0&0&0&0&1&1&0&1&1&1&46&24\\
E4:&0&0&0&1&0&0&1&0&0&1&1&1&46&12\\
E5:&0&0&0&1&0&0&1&1&0&1&1&1&33&24\\
E6:&0&0&0&1&0&0&1&1&1&1&1&1&21&24\\
E7:&0&0&1&1&1&1&1&1&1&1&0&0&20&6\\
E8:&0&0&0&1&0&0&1&1&0&2&1&1&18&24\\
E9:&0&0&0&1&0&1&1&1&1&1&1&0&16&12\\
E10:&0&0&1&1&1&1&1&1&2&1&0&0&12&24\\
\end{tabular}
\end{center}
The representation matrix is:
\begin{center}
\scriptsize
\begin{tabular}{|c|cccccccccc|c|c|}
\hline
&E1&E2&E3&E4&E5&E6&E7&E8&E9&E10&Adj.&Size\\
\hline
E1& 5& 8& 20& 12& 8& 16& 4& 12& 4& 8&97&6\\
E2& 6& 7& 21& 9& 18& 9& 3& 6& 6& 6&91&8\\
E3& 5& 7& 7& 5& 10& 4& 0& 4& 2& 2&46&24\\
E4& 6& 6& 10& 2& 8& 4& 2& 4& 0& 4&46&12\\
E5& 2& 6& 10& 4& 3& 2& 0& 4& 1& 1&33&24\\
E6& 4& 3& 4& 2& 2& 0& 1& 2& 1& 2&21&24\\
E7& 4& 4& 0& 4& 0& 4& 0& 0& 0& 4&20&6\\
E8& 3& 2& 4& 2& 4& 2& 0& 0& 1& 0&18&24\\
E9& 2& 4& 4& 0& 2& 2& 0& 2& 0& 0&16&12\\
E10& 2& 2& 2& 2& 1& 2& 1& 0& 0& 0&12&24\\
\hline
Size&6&8&24&12&24&24&6&24&12&24&&164\\
\hline
\end{tabular}
\end{center}
\subsection{The local graphs}
The complemented local graph are:\\
We consider graph with cardinal inferior or equal to $20$\\
The complemented local graph for orbit $7$ is:
\begin{equation*}
\begin{array}{rrcl}
1&11&:&\,\,5\,\,6\,\,7\,\,8\,\,9\,\,10\,\,11\,\,12\,\,14\,\,15\,\,20\\
2&6&:&\,\,5\,\,7\,\,8\,\,11\,\,13\,\,15\\
3&8&:&\,\,5\,\,6\,\,7\,\,8\,\,10\,\,13\,\,14\,\,18\\
4&3&:&\,\,6\,\,10\,\,11\\
5&11&:&\,\,1\,\,2\,\,3\,\,6\,\,10\,\,11\,\,16\,\,17\,\,18\,\,19\,\,20\\
6&11&:&\,\,1\,\,3\,\,4\,\,5\,\,8\,\,9\,\,11\,\,12\,\,13\,\,15\,\,19\\
7&8&:&\,\,1\,\,2\,\,3\,\,10\,\,11\,\,13\,\,15\,\,19\\
8&6&:&\,\,1\,\,2\,\,3\,\,6\,\,13\,\,18\\
9&2&:&\,\,1\,\,6\\
10&8&:&\,\,1\,\,3\,\,4\,\,5\,\,7\,\,13\,\,15\,\,18\\
11&11&:&\,\,1\,\,2\,\,4\,\,5\,\,6\,\,7\,\,13\,\,14\,\,16\,\,17\,\,18\\
12&2&:&\,\,1\,\,6\\
13&8&:&\,\,2\,\,3\,\,6\,\,7\,\,8\,\,10\,\,11\,\,20\\
14&3&:&\,\,1\,\,3\,\,11\\
15&6&:&\,\,1\,\,2\,\,6\,\,7\,\,10\,\,18\\
16&2&:&\,\,5\,\,11\\
17&2&:&\,\,5\,\,11\\
18&6&:&\,\,3\,\,5\,\,8\,\,10\,\,11\,\,15\\
19&3&:&\,\,5\,\,6\,\,7\\
20&3&:&\,\,1\,\,5\,\,13\\
\end{array}
\end{equation*}
The complemented local graph for orbit $8$ is:
\begin{equation*}
\begin{array}{rrcl}
1&1&:&\,\,2\\
2&8&:&\,\,1\,\,3\,\,4\,\,7\,\,8\,\,9\,\,14\,\,15\\
3&8&:&\,\,2\,\,5\,\,8\,\,11\,\,12\,\,15\,\,17\,\,18\\
4&5&:&\,\,2\,\,7\,\,8\,\,10\,\,17\\
5&6&:&\,\,3\,\,7\,\,9\,\,12\,\,13\,\,17\\
6&3&:&\,\,8\,\,9\,\,11\\
7&6&:&\,\,2\,\,4\,\,5\,\,11\,\,13\,\,14\\
8&7&:&\,\,2\,\,3\,\,4\,\,6\,\,12\,\,13\,\,15\\
9&6&:&\,\,2\,\,5\,\,6\,\,10\,\,13\,\,15\\
10&9&:&\,\,4\,\,9\,\,11\,\,12\,\,13\,\,14\,\,15\,\,16\,\,17\\
11&6&:&\,\,3\,\,6\,\,7\,\,10\,\,13\,\,15\\
12&5&:&\,\,3\,\,5\,\,8\,\,10\,\,14\\
13&6&:&\,\,5\,\,7\,\,8\,\,9\,\,10\,\,11\\
14&5&:&\,\,2\,\,7\,\,10\,\,12\,\,17\\
15&7&:&\,\,2\,\,3\,\,8\,\,9\,\,10\,\,11\,\,16\\
16&2&:&\,\,10\,\,15\\
17&5&:&\,\,3\,\,4\,\,5\,\,10\,\,14\\
18&1&:&\,\,3\\
\end{array}
\end{equation*}
The complemented local graph for orbit $9$ is:
\begin{equation*}
\begin{array}{rrcl}
1&2&:&\,\,3\,\,5\\
2&4&:&\,\,4\,\,5\,\,6\,\,10\\
3&5&:&\,\,1\,\,4\,\,5\,\,9\,\,12\\
4&7&:&\,\,2\,\,3\,\,6\,\,9\,\,11\,\,12\,\,16\\
5&7&:&\,\,1\,\,2\,\,3\,\,8\,\,10\,\,11\,\,15\\
6&4&:&\,\,2\,\,4\,\,8\,\,14\\
7&2&:&\,\,8\,\,9\\
8&7&:&\,\,5\,\,6\,\,7\,\,12\,\,13\,\,14\,\,16\\
9&7&:&\,\,3\,\,4\,\,7\,\,10\,\,12\,\,14\,\,15\\
10&4&:&\,\,2\,\,5\,\,9\,\,14\\
11&2&:&\,\,4\,\,5\\
12&5&:&\,\,3\,\,4\,\,8\,\,9\,\,13\\
13&2&:&\,\,8\,\,12\\
14&4&:&\,\,6\,\,8\,\,9\,\,10\\
15&2&:&\,\,5\,\,9\\
16&2&:&\,\,4\,\,8\\
\end{array}
\end{equation*}
The complemented local graph for orbit $10$ is:
\begin{equation*}
\begin{array}{rrcl}
1&1&:&\,\,6\\
2&2&:&\,\,3\,\,5\\
3&1&:&\,\,2\\
4&0&:&\\
5&2&:&\,\,2\,\,8\\
6&3&:&\,\,1\,\,9\,\,10\\
7&0&:&\\
8&1&:&\,\,5\\
9&1&:&\,\,6\\
10&1&:&\,\,6\\
11&0&:&\\
12&0&:&\\
\end{array}
\end{equation*}
\subsection{scalar product between elements of orbits}
\noindent This subsection made sense only if the representation is orthogonal which is the usual case
Considering Orbit 1
\begin{enumerate}
\item Orbit Size is $6$.
\item Norm of elements is $4$.
\item Possible scalar product : $0$, $1$
\end{enumerate}
Considering Orbit 2
\begin{enumerate}
\item Orbit Size is $8$.
\item Norm of elements is $3$.
\item Possible scalar product : $0$, $1$
\end{enumerate}
Considering Orbit 3
\begin{enumerate}
\item Orbit Size is $24$.
\item Norm of elements is $5$.
\item Possible scalar product : $0$, $1$, $2$, $3$, $4$
\end{enumerate}
Considering Orbit 4
\begin{enumerate}
\item Orbit Size is $12$.
\item Norm of elements is $5$.
\item Possible scalar product : $0$, $1$, $2$, $3$
\end{enumerate}
Considering Orbit 5
\begin{enumerate}
\item Orbit Size is $24$.
\item Norm of elements is $6$.
\item Possible scalar product : $0$, $1$, $2$, $3$, $4$, $5$
\end{enumerate}
Considering Orbit 6
\begin{enumerate}
\item Orbit Size is $24$.
\item Norm of elements is $7$.
\item Possible scalar product : $2$, $3$, $4$, $5$, $6$
\end{enumerate}
Considering Orbit 7
\begin{enumerate}
\item Orbit Size is $6$.
\item Norm of elements is $8$.
\item Possible scalar product : $4$, $5$
\end{enumerate}
Considering Orbit 8
\begin{enumerate}
\item Orbit Size is $24$.
\item Norm of elements is $9$.
\item Possible scalar product : $0$, $1$, $2$, $3$, $4$, $5$, $6$, $7$, $8$
\end{enumerate}
Considering Orbit 9
\begin{enumerate}
\item Orbit Size is $12$.
\item Norm of elements is $7$.
\item Possible scalar product : $2$, $3$, $4$, $5$
\end{enumerate}
Considering Orbit 10
\begin{enumerate}
\item Orbit Size is $24$.
\item Norm of elements is $11$.
\item Possible scalar product : $4$, $5$, $6$, $7$, $8$, $10$
\end{enumerate}
\subsection{invariant group of Orbits}
\noindent The invariant group of $x$ is the set of $g$ such that g(x)=x denoted b $G_x$.\\
If $x$ and $y$ are in the same orbit then the groups $G_x$ and  $G_x$ are conjuguate.\\
Invariant group of Orbit 1
\begin{enumerate}
\item Group Size is $8$.
\item Elements are : (4 3 2 1   reversal), (4 3 1 2   reversal), (3 4 2 1   reversal), (3 4 1 2   reversal), (2 1 4 3  ), (1 2 4 3  ), (2 1 3 4  ), (1 2 3 4  ), 
\end{enumerate}
Invariant group of Orbit 2
\begin{enumerate}
\item Group Size is $6$.
\item Elements are : (3 2 1 4  ), (3 1 2 4  ), (2 3 1 4  ), (1 3 2 4  ), (2 1 3 4  ), (1 2 3 4  ), 
\end{enumerate}
Invariant group of Orbit 3
\begin{enumerate}
\item Group Size is $2$.
\item Elements are : (2 1 3 4  ), (1 2 3 4  ), 
\end{enumerate}
Invariant group of Orbit 4
\begin{enumerate}
\item Group Size is $4$.
\item Elements are : (4 3 2 1   reversal), (4 2 3 1   reversal), (1 3 2 4  ), (1 2 3 4  ), 
\end{enumerate}
Invariant group of Orbit 5
\begin{enumerate}
\item Group Size is $2$.
\item Elements are : (4 3 2 1   reversal), (1 2 3 4  ), 
\end{enumerate}
Invariant group of Orbit 6
\begin{enumerate}
\item Group Size is $2$.
\item Elements are : (1 2 4 3  ), (1 2 3 4  ), 
\end{enumerate}
Invariant group of Orbit 7
\begin{enumerate}
\item Group Size is $8$.
\item Elements are : (4 3 2 1  ), (4 2 3 1  ), (3 4 1 2   reversal), (2 4 1 3   reversal), (3 1 4 2   reversal), (2 1 4 3   reversal), (1 3 2 4  ), (1 2 3 4  ), 
\end{enumerate}
Invariant group of Orbit 8
\begin{enumerate}
\item Group Size is $2$.
\item Elements are : (4 3 2 1   reversal), (1 2 3 4  ), 
\end{enumerate}
Invariant group of Orbit 9
\begin{enumerate}
\item Group Size is $4$.
\item Elements are : (3 4 1 2   reversal), (1 4 3 2  ), (3 2 1 4   reversal), (1 2 3 4  ), 
\end{enumerate}
Invariant group of Orbit 10
\begin{enumerate}
\item Group Size is $2$.
\item Elements are : (2 1 4 3   reversal), (1 2 3 4  ), 
\end{enumerate}
\section{The intrisic informations concerning the graph of facet rays of QMET4.ine}
The graph considered is made of $36$ elements.\\
The group acting on QMET4.ine is ReprS4revers\\
Under the group action we have in fact only $2$ orbits to consider.\\
The diameter of the graph is $2$.\\
The list of orbits is with adjacency and Size info:
\begin{center}
\scriptsize
\begin{tabular}{ccccccccccccc|c|c}
F1:&-1&0&1&0&0&0&0&0&0&0&1&0&28&24\\
F2:&0&0&0&0&0&0&0&0&0&0&0&1&28&12\\
\end{tabular}
\end{center}
The representation matrix is:
\begin{center}
\scriptsize
\begin{tabular}{|c|cc|c|c|}
\hline
&F1&F2&Adj.&Size\\
\hline
F1& 17& 11&28&24\\
F2& 22& 6&28&12\\
\hline
Size&24&12&&36\\
\hline
\end{tabular}
\end{center}
No complemented local graph of size inferior to $20$.
\subsection{scalar product between elements of orbits}
\noindent This subsection made sense only if the representation is orthogonal which is the usual case
Considering Orbit 1
\begin{enumerate}
\item Orbit Size is $24$.
\item Norm of elements is $3$.
\item Possible scalar product : $-2$, $-1$, $0$, $1$
\end{enumerate}
Considering Orbit 2
\begin{enumerate}
\item Orbit Size is $12$.
\item Norm of elements is $1$.
\item Possible scalar product : $0$
\end{enumerate}
\subsection{invariant group of Orbits}
\noindent The invariant group of $x$ is the set of $g$ such that g(x)=x denoted b $G_x$.\\
If $x$ and $y$ are in the same orbit then the groups $G_x$ and  $G_x$ are conjuguate.\\
Invariant group of Orbit 1
\begin{enumerate}
\item Group Size is $2$.
\item Elements are : (2 1 3 4   reversal), (1 2 3 4  ), 
\end{enumerate}
Invariant group of Orbit 2
\begin{enumerate}
\item Group Size is $4$.
\item Elements are : (2 1 4 3   reversal), (1 2 4 3   reversal), (2 1 3 4  ), (1 2 3 4  ), 
\end{enumerate}
\section{The two incidence matrix}
Incidence between Orbit of Extreme rays and Orbits of Facets
\begin{equation*}
\begin{array}{|c|cc|c|}
\hline
&F1&F2&Inc\\
\hline
E1&16&8&24\\
E2&18&9&27\\
E3&14&7&21\\
E4&14&7&21\\
E5&12&6&18\\
E6&10&5&15\\
E7&8&4&12\\
E8&10&6&16\\
E9&10&5&15\\
E10&8&4&12\\
\hline
\end{array}
\end{equation*}
Incidence between Orbits of Facets and Orbits of Extreme Rays
\begin{equation*}
\begin{array}{|c|cccccccccc|c|}
\hline
&E1&E2&E3&E4&E5&E6&E7&E8&E9&E10&Inc\\
\hline
F1&4&6&14&7&12&10&2&10&5&8&78\\
F2&4&6&14&7&12&10&2&12&5&8&80\\
\hline
\end{array}
\end{equation*}
\end{document}
