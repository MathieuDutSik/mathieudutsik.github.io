\documentclass[12pt]{article}
\usepackage{graphicx,amssymb,amsmath,vmargin,multicol}
\usepackage[english]{babel}
\title{The Cone OMCUT4}
\setpapersize{custom}{21cm}{29.7cm}
\setmarginsrb{1cm}{1cm}{1cm}{2cm}{0pt}{0pt}{0pt}{0pt}
\begin{document}
\maketitle
\section{The intrisic informations concerning the graph of extreme rays of OMCUT4.ext}
The graph considered is made of $74$ elements.\\
The group acting on OMCUT4.ext is ReprS4revers\\
Under the group action we have in fact only $5$ orbits to consider.\\
The diameter of the graph is $2$.\\
The list of orbits is with adjacency and Size info:
\begin{center}
\scriptsize
\begin{tabular}{ccccccccccccc|c|c}
E1:&0&0&0&0&0&0&0&0&0&1&1&1&61&8\\
E2:&0&0&0&0&0&0&1&1&0&1&1&0&53&6\\
E3:&0&0&0&0&0&0&1&1&0&1&1&1&44&24\\
E4:&0&0&0&1&0&0&1&0&0&1&1&1&40&12\\
E5:&0&0&0&1&0&0&1&1&0&1&1&1&34&24\\
\end{tabular}
\end{center}
The representation matrix is:
\begin{center}
\scriptsize
\begin{tabular}{|c|ccccc|c|c|}
\hline
&E1&E2&E3&E4&E5&Adj.&Size\\
\hline
E1& 7& 6& 21& 9& 18&61&8\\
E2& 8& 5& 20& 12& 8&53&6\\
E3& 7& 5& 15& 7& 10&44&24\\
E4& 6& 6& 14& 6& 8&40&12\\
E5& 6& 2& 10& 4& 12&34&24\\
\hline
Size&8&6&24&12&24&&74\\
\hline
\end{tabular}
\end{center}
No complemented local graph of size inferior to $20$.
\subsection{scalar product between elements of orbits}
\noindent This subsection made sense only if the representation is orthogonal which is the usual case
Considering Orbit 1
\begin{enumerate}
\item Orbit Size is $8$.
\item Norm of elements is $3$.
\item Possible scalar product : $0$, $1$
\end{enumerate}
Considering Orbit 2
\begin{enumerate}
\item Orbit Size is $6$.
\item Norm of elements is $4$.
\item Possible scalar product : $0$, $1$
\end{enumerate}
Considering Orbit 3
\begin{enumerate}
\item Orbit Size is $24$.
\item Norm of elements is $5$.
\item Possible scalar product : $0$, $1$, $2$, $3$, $4$
\end{enumerate}
Considering Orbit 4
\begin{enumerate}
\item Orbit Size is $12$.
\item Norm of elements is $5$.
\item Possible scalar product : $0$, $1$, $2$, $3$
\end{enumerate}
Considering Orbit 5
\begin{enumerate}
\item Orbit Size is $24$.
\item Norm of elements is $6$.
\item Possible scalar product : $0$, $1$, $2$, $3$, $4$, $5$
\end{enumerate}
\subsection{invariant group of Orbits}
\noindent The invariant group of $x$ is the set of $g$ such that g(x)=x denoted b $G_x$.\\
If $x$ and $y$ are in the same orbit then the groups $G_x$ and  $G_x$ are conjuguate.\\
Invariant group of Orbit 1
\begin{enumerate}
\item Group Size is $6$.
\item Elements are : (3 2 1 4  ), (3 1 2 4  ), (2 3 1 4  ), (1 3 2 4  ), (2 1 3 4  ), (1 2 3 4  ), 
\end{enumerate}
Invariant group of Orbit 2
\begin{enumerate}
\item Group Size is $8$.
\item Elements are : (4 3 2 1   reversal), (4 3 1 2   reversal), (3 4 2 1   reversal), (3 4 1 2   reversal), (2 1 4 3  ), (1 2 4 3  ), (2 1 3 4  ), (1 2 3 4  ), 
\end{enumerate}
Invariant group of Orbit 3
\begin{enumerate}
\item Group Size is $2$.
\item Elements are : (2 1 3 4  ), (1 2 3 4  ), 
\end{enumerate}
Invariant group of Orbit 4
\begin{enumerate}
\item Group Size is $4$.
\item Elements are : (4 3 2 1   reversal), (4 2 3 1   reversal), (1 3 2 4  ), (1 2 3 4  ), 
\end{enumerate}
Invariant group of Orbit 5
\begin{enumerate}
\item Group Size is $2$.
\item Elements are : (4 3 2 1   reversal), (1 2 3 4  ), 
\end{enumerate}
\section{The intrisic informations concerning the graph of facet rays of OMCUT4.ine}
The graph considered is made of $72$ elements.\\
The group acting on OMCUT4.ine is ReprS4revers\\
Under the group action we have in fact only $4$ orbits to consider.\\
The diameter of the graph is $2$.\\
The list of orbits is with adjacency and Size info:
\begin{center}
\scriptsize
\begin{tabular}{ccccccccccccc|c|c}
F1:&0&0&0&0&0&0&0&0&0&0&0&1&48&12\\
F2:&-1&0&1&0&0&0&0&0&0&0&1&0&41&24\\
F3:&-1&0&1&-1&0&1&1&1&-1&0&0&1&24&12\\
F4:&-1&0&1&-1&0&1&1&1&-1&1&1&0&16&24\\
\end{tabular}
\end{center}
The representation matrix is:
\begin{center}
\scriptsize
\begin{tabular}{|c|cccc|c|c|}
\hline
&F1&F2&F3&F4&Adj.&Size\\
\hline
F1& 6& 22& 12& 8&48&12\\
F2& 11& 17& 5& 8&41&24\\
F3& 12& 10& 0& 2&24&12\\
F4& 4& 8& 1& 3&16&24\\
\hline
Size&12&24&12&24&&72\\
\hline
\end{tabular}
\end{center}
\subsection{The local graphs}
The complemented local graph are:\\
We consider graph with cardinal inferior or equal to $20$\\
The complemented local graph for orbit $4$ is:
\begin{equation*}
\begin{array}{rrcl}
1&5&:&\,\,5\,\,6\,\,7\,\,8\,\,9\\
2&1&:&\,\,10\\
3&2&:&\,\,5\,\,10\\
4&5&:&\,\,5\,\,6\,\,7\,\,11\,\,13\\
5&4&:&\,\,1\,\,3\,\,4\,\,10\\
6&4&:&\,\,1\,\,4\,\,10\,\,13\\
7&3&:&\,\,1\,\,4\,\,10\\
8&1&:&\,\,1\\
9&1&:&\,\,1\\
10&5&:&\,\,2\,\,3\,\,5\,\,6\,\,7\\
11&1&:&\,\,4\\
12&0&:&\\
13&2&:&\,\,4\,\,6\\
14&0&:&\\
15&0&:&\\
16&0&:&\\
\end{array}
\end{equation*}
\subsection{scalar product between elements of orbits}
\noindent This subsection made sense only if the representation is orthogonal which is the usual case
Considering Orbit 1
\begin{enumerate}
\item Orbit Size is $12$.
\item Norm of elements is $1$.
\item Possible scalar product : $0$
\end{enumerate}
Considering Orbit 2
\begin{enumerate}
\item Orbit Size is $24$.
\item Norm of elements is $3$.
\item Possible scalar product : $-2$, $-1$, $0$, $1$
\end{enumerate}
Considering Orbit 3
\begin{enumerate}
\item Orbit Size is $12$.
\item Norm of elements is $8$.
\item Possible scalar product : $-3$, $0$, $1$, $2$
\end{enumerate}
Considering Orbit 4
\begin{enumerate}
\item Orbit Size is $24$.
\item Norm of elements is $9$.
\item Possible scalar product : $-3$, $-2$, $-1$, $6$, $7$
\end{enumerate}
\subsection{invariant group of Orbits}
\noindent The invariant group of $x$ is the set of $g$ such that g(x)=x denoted b $G_x$.\\
If $x$ and $y$ are in the same orbit then the groups $G_x$ and  $G_x$ are conjuguate.\\
Invariant group of Orbit 1
\begin{enumerate}
\item Group Size is $4$.
\item Elements are : (2 1 4 3   reversal), (1 2 4 3   reversal), (2 1 3 4  ), (1 2 3 4  ), 
\end{enumerate}
Invariant group of Orbit 2
\begin{enumerate}
\item Group Size is $2$.
\item Elements are : (2 1 3 4   reversal), (1 2 3 4  ), 
\end{enumerate}
Invariant group of Orbit 3
\begin{enumerate}
\item Group Size is $4$.
\item Elements are : (2 1 4 3   reversal), (1 2 4 3   reversal), (2 1 3 4  ), (1 2 3 4  ), 
\end{enumerate}
Invariant group of Orbit 4
\begin{enumerate}
\item Group Size is $2$.
\item Elements are : (2 1 3 4  ), (1 2 3 4  ), 
\end{enumerate}
\section{The two incidence matrix}
Incidence between Orbit of Extreme rays and Orbits of Facets
\begin{equation*}
\begin{array}{|c|cccc|c|}
\hline
&F1&F2&F3&F4&Inc\\
\hline
E1&9&18&6&9&42\\
E2&8&16&8&16&48\\
E3&7&14&4&4&29\\
E4&7&14&4&8&33\\
E5&6&12&4&2&24\\
\hline
\end{array}
\end{equation*}
Incidence between Orbits of Facets and Orbits of Extreme Rays
\begin{equation*}
\begin{array}{|c|ccccc|c|}
\hline
&E1&E2&E3&E4&E5&Inc\\
\hline
F1&6&4&14&7&12&43\\
F2&6&4&14&7&12&43\\
F3&4&4&8&4&8&28\\
F4&3&4&4&4&2&17\\
\hline
\end{array}
\end{equation*}
\end{document}
