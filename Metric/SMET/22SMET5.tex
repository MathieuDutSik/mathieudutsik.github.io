\documentclass[12pt]{article}
\usepackage{graphicx,amssymb,amsmath,vmargin,multicol}
\usepackage[english]{babel}
\title{The Cone 22SMET5}
\setpapersize{custom}{21cm}{29.7cm}
\setmarginsrb{1cm}{1cm}{1cm}{2cm}{0pt}{0pt}{0pt}{0pt}
\begin{document}
\maketitle
\section{The intrisic informations concerning the graph of extreme rays of 22SMET5.ext}
The graph considered is made of $132$ elements.\\
The group acting on 22SMET5.ext is Group2hmet5\\
Under the group action we have in fact only $6$ orbits to consider.\\
The diameter of the graph is $2$.\\
The list of orbits is with adjacency and Size info:
\begin{center}
\scriptsize
\begin{tabular}{ccccccccccc|c|c}
E1:&0&0&1&0&1&1&0&1&1&1&92&5\\
E2:&0&2&2&2&2&1&2&2&1&1&32&10\\
E3:&0&1&1&1&1&0&1&1&1&1&28&15\\
E4:&1&1&2&2&1&2&2&2&1&1&25&12\\
E5:&1&2&4&3&3&2&3&4&4&1&13&60\\
E6:&0&4&4&4&4&2&4&4&2&5&13&30\\
\end{tabular}
\end{center}
The representation matrix is:
\begin{center}
\scriptsize
\begin{tabular}{|c|cccccc|c|c|}
\hline
&E1&E2&E3&E4&E5&E6&Adj.&Size\\
\hline
E1& 4& 10& 12& 12& 36& 18&92&5\\
E2& 5& 0& 3& 6& 12& 6&32&10\\
E3& 4& 2& 2& 4& 12& 4&28&15\\
E4& 5& 5& 5& 0& 5& 5&25&12\\
E5& 3& 2& 3& 1& 2& 2&13&60\\
E6& 3& 2& 2& 2& 4& 0&13&30\\
\hline
Size&5&10&15&12&60&30&&132\\
\hline
\end{tabular}
\end{center}
\subsection{The local graphs}
The complemented local graph are:\\
We consider graph with cardinal inferior or equal to $20$\\
The complemented local graph for orbit $5$ is:
\begin{equation*}
\begin{array}{rrcl}
1&6&:&\,\,2\,\,3\,\,5\,\,7\,\,8\,\,12\\
2&6&:&\,\,1\,\,5\,\,6\,\,10\,\,11\,\,12\\
3&5&:&\,\,1\,\,6\,\,7\,\,10\,\,12\\
4&0&:&\\
5&2&:&\,\,1\,\,2\\
6&5&:&\,\,2\,\,3\,\,8\,\,11\,\,12\\
7&4&:&\,\,1\,\,3\,\,10\,\,11\\
8&5&:&\,\,1\,\,6\,\,9\,\,10\,\,11\\
9&3&:&\,\,8\,\,10\,\,12\\
10&5&:&\,\,2\,\,3\,\,7\,\,8\,\,9\\
11&4&:&\,\,2\,\,6\,\,7\,\,8\\
12&5&:&\,\,1\,\,2\,\,3\,\,6\,\,9\\
13&0&:&\\
\end{array}
\end{equation*}
The complemented local graph for orbit $6$ is:
\begin{equation*}
\begin{array}{rrcl}
1&3&:&\,\,4\,\,6\,\,7\\
2&4&:&\,\,4\,\,5\,\,8\,\,9\\
3&5&:&\,\,5\,\,6\,\,7\,\,8\,\,12\\
4&5&:&\,\,1\,\,2\,\,7\,\,8\,\,12\\
5&4&:&\,\,2\,\,3\,\,7\,\,10\\
6&3&:&\,\,1\,\,3\,\,8\\
7&5&:&\,\,1\,\,3\,\,4\,\,5\,\,13\\
8&5&:&\,\,2\,\,3\,\,4\,\,6\,\,13\\
9&2&:&\,\,2\,\,10\\
10&2&:&\,\,5\,\,9\\
11&0&:&\\
12&2&:&\,\,3\,\,4\\
13&2&:&\,\,7\,\,8\\
\end{array}
\end{equation*}
\subsection{scalar product between elements of orbits}
\noindent This subsection made sense only if the representation is orthogonal which is the usual case
Considering Orbit 1
\begin{enumerate}
\item Orbit Size is $5$.
\item Norm of elements is $6$.
\item Possible scalar product : $3$
\end{enumerate}
Considering Orbit 2
\begin{enumerate}
\item Orbit Size is $10$.
\item Norm of elements is $27$.
\item Possible scalar product : $21$, $24$
\end{enumerate}
Considering Orbit 3
\begin{enumerate}
\item Orbit Size is $15$.
\item Norm of elements is $8$.
\item Possible scalar product : $6$, $7$
\end{enumerate}
Considering Orbit 4
\begin{enumerate}
\item Orbit Size is $12$.
\item Norm of elements is $25$.
\item Possible scalar product : $20$, $22$, $23$
\end{enumerate}
Considering Orbit 5
\begin{enumerate}
\item Orbit Size is $60$.
\item Norm of elements is $85$.
\item Possible scalar product : $67$, $68$, $70$, $71$, $73$, $74$, $76$, $77$, $79$, $80$, $83$
\end{enumerate}
Considering Orbit 6
\begin{enumerate}
\item Orbit Size is $30$.
\item Norm of elements is $129$.
\item Possible scalar product : $96$, $102$, $105$, $108$, $120$
\end{enumerate}
\subsection{invariant group of Orbits}
\noindent The invariant group of $x$ is the set of $g$ such that $g(x)=x$ denoted by $G_x$.\\
If $x$ and $y$ are in the same orbit then the groups $G_x$ and  $G_y$ are conjuguate.\\
For any element $x$ we denote $O_x$ its Orbit, we have $|O_x|\times |G_x|=|G|$
We print only nontrivial groups
Invariant group of Orbit $1$:
\begin{enumerate}
\item Group Size is $24$.
\item Elements are : (4 3 2 1 5  ), (4 3 1 2 5  ), (4 2 3 1 5  ), (4 1 3 2 5  ), (4 2 1 3 5  ), (4 1 2 3 5  ), (3 4 2 1 5  ), (3 4 1 2 5  ), (2 4 3 1 5  ), (1 4 3 2 5  ), (2 4 1 3 5  ), (1 4 2 3 5  ), (3 2 4 1 5  ), (3 1 4 2 5  ), (2 3 4 1 5  ), (1 3 4 2 5  ), (2 1 4 3 5  ), (1 2 4 3 5  ), (3 2 1 4 5  ), (3 1 2 4 5  ), (2 3 1 4 5  ), (1 3 2 4 5  ), (2 1 3 4 5  ), (1 2 3 4 5  )
\end{enumerate}
Invariant group of Orbit $2$:
\begin{enumerate}
\item Group Size is $12$.
\item Elements are : (3 2 1 5 4  ), (3 1 2 5 4  ), (2 3 1 5 4  ), (1 3 2 5 4  ), (2 1 3 5 4  ), (1 2 3 5 4  ), (3 2 1 4 5  ), (3 1 2 4 5  ), (2 3 1 4 5  ), (1 3 2 4 5  ), (2 1 3 4 5  ), (1 2 3 4 5  )
\end{enumerate}
Invariant group of Orbit $3$:
\begin{enumerate}
\item Group Size is $8$.
\item Elements are : (1 5 4 3 2  ), (1 5 4 2 3  ), (1 4 5 3 2  ), (1 4 5 2 3  ), (1 3 2 5 4  ), (1 2 3 5 4  ), (1 3 2 4 5  ), (1 2 3 4 5  )
\end{enumerate}
Invariant group of Orbit $4$:
\begin{enumerate}
\item Group Size is $10$.
\item Elements are : (5 4 3 2 1  ), (5 3 4 1 2  ), (4 5 2 3 1  ), (3 5 1 4 2  ), (4 2 5 1 3  ), (3 1 5 2 4  ), (2 4 1 5 3  ), (1 3 2 5 4  ), (2 1 4 3 5  ), (1 2 3 4 5  )
\end{enumerate}
Invariant group of Orbit $5$:
\begin{enumerate}
\item Group Size is $2$.
\item Elements are : (4 5 3 1 2  ), (1 2 3 4 5  )
\end{enumerate}
Invariant group of Orbit $6$:
\begin{enumerate}
\item Group Size is $4$.
\item Elements are : (2 1 3 5 4  ), (1 2 3 5 4  ), (2 1 3 4 5  ), (1 2 3 4 5  )
\end{enumerate}
\section{The intrisic informations concerning the graph of facet rays of 22SMET5.ine}
The graph considered is made of $20$ elements.\\
The group acting on 22SMET5.ine is Group2hmet5\\
Under the group action we have in fact only $1$ orbits to consider.\\
The diameter of the graph is $1$.\\
The list of orbits is with adjacency and Size info:
\begin{center}
\scriptsize
\begin{tabular}{ccccccccccc|c|c}
F1:&-2&0&1&0&1&0&0&1&0&0&19&20\\
\end{tabular}
\end{center}
The representation matrix is:
\begin{center}
\scriptsize
\begin{tabular}{|c|c|c|c|}
\hline
&F1&Adj.&Size\\
\hline
F1& 19&19&20\\
\hline
Size&20&&20\\
\hline
\end{tabular}
\end{center}
\subsection{The local graphs}
The complemented local graph are:\\
We consider graph with cardinal inferior or equal to $20$\\
The complemented local graph for orbit $1$ is:
\begin{equation*}
\begin{array}{rrcl}
1&0&:&\\
2&0&:&\\
3&0&:&\\
4&0&:&\\
5&0&:&\\
6&0&:&\\
7&0&:&\\
8&0&:&\\
9&0&:&\\
10&0&:&\\
11&0&:&\\
12&0&:&\\
13&0&:&\\
14&0&:&\\
15&0&:&\\
16&0&:&\\
17&0&:&\\
18&0&:&\\
19&0&:&\\
\end{array}
\end{equation*}
\subsection{scalar product between elements of orbits}
\noindent This subsection made sense only if the representation is orthogonal which is the usual case
Considering Orbit 1
\begin{enumerate}
\item Orbit Size is $20$.
\item Norm of elements is $7$.
\item Possible scalar product : $-2$, $1$, $4$
\end{enumerate}
\subsection{invariant group of Orbits}
\noindent The invariant group of $x$ is the set of $g$ such that $g(x)=x$ denoted by $G_x$.\\
If $x$ and $y$ are in the same orbit then the groups $G_x$ and  $G_y$ are conjuguate.\\
For any element $x$ we denote $O_x$ its Orbit, we have $|O_x|\times |G_x|=|G|$
We print only nontrivial groups
Invariant group of Orbit $1$:
\begin{enumerate}
\item Group Size is $6$.
\item Elements are : (3 2 1 4 5  ), (3 1 2 4 5  ), (2 3 1 4 5  ), (1 3 2 4 5  ), (2 1 3 4 5  ), (1 2 3 4 5  )
\end{enumerate}
\section{The two incidence matrix}
Incidence between Orbit of Extreme rays and Orbits of Facets
\begin{equation*}
\begin{array}{|c|c|c|}
\hline
&F1&Inc\\
\hline
E1&16&16\\
E2&12&12\\
E3&12&12\\
E4&10&10\\
E5&10&10\\
E6&10&10\\
\hline
\end{array}
\end{equation*}
Incidence between Orbits of Facets and Orbits of Extreme Rays
\begin{equation*}
\begin{array}{|c|cccccc|c|}
\hline
&E1&E2&E3&E4&E5&E6&Inc\\
\hline
F1&4&6&9&6&30&15&70\\
\hline
\end{array}
\end{equation*}
\end{document}
