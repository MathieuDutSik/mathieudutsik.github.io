\documentclass[12pt]{article}
\usepackage{graphicx,amssymb,amsmath,vmargin,multicol}
\usepackage[english]{babel}
\title{The Cone 22BRO5}
\setpapersize{custom}{21cm}{29.7cm}
\setmarginsrb{1cm}{1cm}{1cm}{2cm}{0pt}{0pt}{0pt}{0pt}
\begin{document}
\maketitle
\section{The intrisic informations concerning the graph of extreme rays of 22BRO5.ext}
The graph considered is made of $20$ elements.\\
The group acting on 22BRO5.ext is Group2hmet5\\
Under the group action we have in fact only $2$ orbits to consider.\\
The diameter of the graph is $1$.\\
The list of orbits is with adjacency and Size info:
\begin{center}
\scriptsize
\begin{tabular}{ccccccccccc|c|c}
E1:&0&1&1&1&1&0&1&1&1&1&19&15\\
E2:&0&0&1&0&1&1&0&1&1&1&19&5\\
\end{tabular}
\end{center}
The representation matrix is:
\begin{center}
\scriptsize
\begin{tabular}{|c|cc|c|c|}
\hline
&E1&E2&Adj.&Size\\
\hline
E1& 14& 5&19&15\\
E2& 15& 4&19&5\\
\hline
Size&15&5&&20\\
\hline
\end{tabular}
\end{center}
\subsection{The local graphs}
The complemented local graph are:\\
We consider graph with cardinal inferior or equal to $20$\\
The complemented local graph for orbit $1$ is:
\begin{equation*}
\begin{array}{rrcl}
1&0&:&\\
2&0&:&\\
3&0&:&\\
4&0&:&\\
5&0&:&\\
6&0&:&\\
7&0&:&\\
8&0&:&\\
9&0&:&\\
10&0&:&\\
11&0&:&\\
12&0&:&\\
13&0&:&\\
14&0&:&\\
15&0&:&\\
16&0&:&\\
17&0&:&\\
18&0&:&\\
19&0&:&\\
\end{array}
\end{equation*}
The complemented local graph for orbit $2$ is:
\begin{equation*}
\begin{array}{rrcl}
1&0&:&\\
2&0&:&\\
3&0&:&\\
4&0&:&\\
5&0&:&\\
6&0&:&\\
7&0&:&\\
8&0&:&\\
9&0&:&\\
10&0&:&\\
11&0&:&\\
12&0&:&\\
13&0&:&\\
14&0&:&\\
15&0&:&\\
16&0&:&\\
17&0&:&\\
18&0&:&\\
19&0&:&\\
\end{array}
\end{equation*}
\subsection{scalar product between elements of orbits}
\noindent This subsection made sense only if the representation is orthogonal which is the usual case
Considering Orbit 1
\begin{enumerate}
\item Orbit Size is $15$.
\item Norm of elements is $8$.
\item Possible scalar product : $6$, $7$
\end{enumerate}
Considering Orbit 2
\begin{enumerate}
\item Orbit Size is $5$.
\item Norm of elements is $6$.
\item Possible scalar product : $3$
\end{enumerate}
\subsection{invariant group of Orbits}
\noindent The invariant group of $x$ is the set of $g$ such that g(x)=x denoted b $G_x$.\\
If $x$ and $y$ are in the same orbit then the groups $G_x$ and  $G_x$ are conjuguate.\\
Invariant group of Orbit 1
\begin{enumerate}
\item Group Size is $8$.
\item Elements are : (1 5 4 3 2  ), (1 5 4 2 3  ), (1 4 5 3 2  ), (1 4 5 2 3  ), (1 3 2 5 4  ), (1 2 3 5 4  ), (1 3 2 4 5  ), (1 2 3 4 5  ), 
\end{enumerate}
Invariant group of Orbit 2
\begin{enumerate}
\item Group Size is $24$.
\item Elements are : (4 3 2 1 5  ), (4 3 1 2 5  ), (4 2 3 1 5  ), (4 1 3 2 5  ), (4 2 1 3 5  ), (4 1 2 3 5  ), (3 4 2 1 5  ), (3 4 1 2 5  ), (2 4 3 1 5  ), (1 4 3 2 5  ), (2 4 1 3 5  ), (1 4 2 3 5  ), (3 2 4 1 5  ), (3 1 4 2 5  ), (2 3 4 1 5  ), (1 3 4 2 5  ), (2 1 4 3 5  ), (1 2 4 3 5  ), (3 2 1 4 5  ), (3 1 2 4 5  ), (2 3 1 4 5  ), (1 3 2 4 5  ), (2 1 3 4 5  ), (1 2 3 4 5  ), 
\end{enumerate}
\section{The intrisic informations concerning the graph of facet rays of 22BRO5.ine}
The graph considered is made of $220$ elements.\\
The group acting on 22BRO5.ine is Group2hmet5\\
Under the group action we have in fact only $6$ orbits to consider.\\
The diameter of the graph is $3$.\\
The list of orbits is with adjacency and Size info:
\begin{center}
\scriptsize
\begin{tabular}{ccccccccccc|c|c}
F1:&-2&0&1&0&1&0&0&1&0&0&61&20\\
F2:&-1&-3&1&1&1&1&1&1&1&-1&22&60\\
F3:&-3&-3&3&-3&3&3&1&1&1&1&15&20\\
F4:&-1&-1&1&1&1&1&1&1&1&-3&13&30\\
F5:&-3&-3&3&1&1&1&1&3&3&-3&9&60\\
F6:&-1&-1&3&5&3&3&5&3&3&-15&9&30\\
\end{tabular}
\end{center}
The representation matrix is:
\begin{center}
\scriptsize
\begin{tabular}{|c|cccccc|c|c|}
\hline
&F1&F2&F3&F4&F5&F6&Adj.&Size\\
\hline
F1& 13& 21& 6& 6& 9& 6&61&20\\
F2& 7& 6& 2& 3& 2& 2&22&60\\
F3& 6& 6& 0& 0& 3& 0&15&20\\
F4& 4& 6& 0& 0& 2& 1&13&30\\
F5& 3& 2& 1& 1& 2& 0&9&60\\
F6& 4& 4& 0& 1& 0& 0&9&30\\
\hline
Size&20&60&20&30&60&30&&220\\
\hline
\end{tabular}
\end{center}
\subsection{The local graphs}
The complemented local graph are:\\
We consider graph with cardinal inferior or equal to $20$\\
The complemented local graph for orbit $3$ is:
\begin{equation*}
\begin{array}{rrcl}
1&10&:&\,\,2\,\,3\,\,4\,\,5\,\,6\,\,8\,\,9\,\,12\,\,14\,\,15\\
2&10&:&\,\,1\,\,3\,\,4\,\,5\,\,6\,\,10\,\,11\,\,12\,\,13\,\,15\\
3&10&:&\,\,1\,\,2\,\,4\,\,7\,\,8\,\,9\,\,10\,\,11\,\,12\,\,15\\
4&6&:&\,\,1\,\,2\,\,3\,\,7\,\,9\,\,10\\
5&8&:&\,\,1\,\,2\,\,7\,\,8\,\,9\,\,10\,\,11\,\,15\\
6&8&:&\,\,1\,\,2\,\,7\,\,8\,\,9\,\,10\,\,11\,\,12\\
7&4&:&\,\,3\,\,4\,\,5\,\,6\\
8&8&:&\,\,1\,\,3\,\,5\,\,6\,\,10\,\,11\,\,13\,\,15\\
9&8&:&\,\,1\,\,3\,\,4\,\,5\,\,6\,\,10\,\,11\,\,13\\
10&8&:&\,\,2\,\,3\,\,4\,\,5\,\,6\,\,8\,\,9\,\,14\\
11&8&:&\,\,2\,\,3\,\,5\,\,6\,\,8\,\,9\,\,12\,\,14\\
12&6&:&\,\,1\,\,2\,\,3\,\,6\,\,11\,\,13\\
13&4&:&\,\,2\,\,8\,\,9\,\,12\\
14&4&:&\,\,1\,\,10\,\,11\,\,15\\
15&6&:&\,\,1\,\,2\,\,3\,\,5\,\,8\,\,14\\
\end{array}
\end{equation*}
The complemented local graph for orbit $4$ is:
\begin{equation*}
\begin{array}{rrcl}
1&1&:&\,\,8\\
2&6&:&\,\,4\,\,5\,\,8\,\,9\,\,10\,\,11\\
3&6&:&\,\,4\,\,7\,\,8\,\,10\,\,11\,\,12\\
4&7&:&\,\,2\,\,3\,\,5\,\,6\,\,7\,\,8\,\,13\\
5&6&:&\,\,2\,\,4\,\,8\,\,10\,\,11\,\,12\\
6&6&:&\,\,4\,\,8\,\,9\,\,10\,\,11\,\,12\\
7&6&:&\,\,3\,\,4\,\,8\,\,9\,\,10\,\,11\\
8&7&:&\,\,1\,\,2\,\,3\,\,4\,\,5\,\,6\,\,7\\
9&3&:&\,\,2\,\,6\,\,7\\
10&5&:&\,\,2\,\,3\,\,5\,\,6\,\,7\\
11&5&:&\,\,2\,\,3\,\,5\,\,6\,\,7\\
12&3&:&\,\,3\,\,5\,\,6\\
13&1&:&\,\,4\\
\end{array}
\end{equation*}
The complemented local graph for orbit $5$ is:
\begin{equation*}
\begin{array}{rrcl}
1&3&:&\,\,2\,\,3\,\,4\\
2&6&:&\,\,1\,\,3\,\,4\,\,6\,\,7\,\,8\\
3&6&:&\,\,1\,\,2\,\,4\,\,5\,\,6\,\,7\\
4&4&:&\,\,1\,\,2\,\,3\,\,9\\
5&1&:&\,\,3\\
6&2&:&\,\,2\,\,3\\
7&2&:&\,\,2\,\,3\\
8&1&:&\,\,2\\
9&1&:&\,\,4\\
\end{array}
\end{equation*}
The complemented local graph for orbit $6$ is:
\begin{equation*}
\begin{array}{rrcl}
1&0&:&\\
2&2&:&\,\,5\,\,8\\
3&2&:&\,\,7\,\,8\\
4&2&:&\,\,6\,\,7\\
5&2&:&\,\,2\,\,7\\
6&2&:&\,\,4\,\,8\\
7&3&:&\,\,3\,\,4\,\,5\\
8&3&:&\,\,2\,\,3\,\,6\\
9&0&:&\\
\end{array}
\end{equation*}
\subsection{scalar product between elements of orbits}
\noindent This subsection made sense only if the representation is orthogonal which is the usual case
Considering Orbit 1
\begin{enumerate}
\item Orbit Size is $20$.
\item Norm of elements is $7$.
\item Possible scalar product : $-2$, $1$, $4$
\end{enumerate}
Considering Orbit 2
\begin{enumerate}
\item Orbit Size is $60$.
\item Norm of elements is $18$.
\item Possible scalar product : $-6$, $-2$, $2$, $6$, $10$, $14$
\end{enumerate}
Considering Orbit 3
\begin{enumerate}
\item Orbit Size is $20$.
\item Norm of elements is $58$.
\item Possible scalar product : $-14$, $-2$, $10$, $34$
\end{enumerate}
Considering Orbit 4
\begin{enumerate}
\item Orbit Size is $30$.
\item Norm of elements is $18$.
\item Possible scalar product : $-6$, $-2$, $2$, $10$, $14$
\end{enumerate}
Considering Orbit 5
\begin{enumerate}
\item Orbit Size is $60$.
\item Norm of elements is $58$.
\item Possible scalar product : $-30$, $-22$, $-18$, $-14$, $-2$, $2$, $10$, $14$, $18$, $26$, $42$, $50$
\end{enumerate}
Considering Orbit 6
\begin{enumerate}
\item Orbit Size is $30$.
\item Norm of elements is $322$.
\item Possible scalar product : $-110$, $-86$, $-74$, $-38$, $-26$, $22$, $106$, $298$
\end{enumerate}
\subsection{invariant group of Orbits}
\noindent The invariant group of $x$ is the set of $g$ such that g(x)=x denoted b $G_x$.\\
If $x$ and $y$ are in the same orbit then the groups $G_x$ and  $G_x$ are conjuguate.\\
Invariant group of Orbit 1
\begin{enumerate}
\item Group Size is $6$.
\item Elements are : (3 2 1 4 5  ), (3 1 2 4 5  ), (2 3 1 4 5  ), (1 3 2 4 5  ), (2 1 3 4 5  ), (1 2 3 4 5  ), 
\end{enumerate}
Invariant group of Orbit 2
\begin{enumerate}
\item Group Size is $2$.
\item Elements are : (2 1 3 4 5  ), (1 2 3 4 5  ), 
\end{enumerate}
Invariant group of Orbit 3
\begin{enumerate}
\item Group Size is $6$.
\item Elements are : (1 4 3 2 5  ), (1 4 2 3 5  ), (1 3 4 2 5  ), (1 2 4 3 5  ), (1 3 2 4 5  ), (1 2 3 4 5  ), 
\end{enumerate}
Invariant group of Orbit 4
\begin{enumerate}
\item Group Size is $4$.
\item Elements are : (2 1 4 3 5  ), (1 2 4 3 5  ), (2 1 3 4 5  ), (1 2 3 4 5  ), 
\end{enumerate}
Invariant group of Orbit 5
\begin{enumerate}
\item Group Size is $2$.
\item Elements are : (1 2 4 3 5  ), (1 2 3 4 5  ), 
\end{enumerate}
Invariant group of Orbit 6
\begin{enumerate}
\item Group Size is $4$.
\item Elements are : (2 1 4 3 5  ), (1 2 4 3 5  ), (2 1 3 4 5  ), (1 2 3 4 5  ), 
\end{enumerate}
\section{The two incidence matrix}
Incidence between Orbit of Extreme rays and Orbits of Facets
\begin{equation*}
\begin{array}{|c|cccccc|c|}
\hline
&F1&F2&F3&F4&F5&F6&Inc\\
\hline
E1&12&32&8&16&24&12&104\\
E2&16&36&16&12&36&18&134\\
\hline
\end{array}
\end{equation*}
Incidence between Orbits of Facets and Orbits of Extreme Rays
\begin{equation*}
\begin{array}{|c|cc|c|}
\hline
&E1&E2&Inc\\
\hline
F1&9&4&13\\
F2&8&3&11\\
F3&6&4&10\\
F4&8&2&10\\
F5&6&3&9\\
F6&6&3&9\\
\hline
\end{array}
\end{equation*}
\end{document}
