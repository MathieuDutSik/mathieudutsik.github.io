\documentclass[12pt]{article}
\usepackage{graphicx,amssymb,amsmath,vmargin,multicol}
\usepackage[english]{babel}
\title{The Cone 33BRO6}
\setpapersize{custom}{21cm}{29.7cm}
\setmarginsrb{1cm}{1cm}{1cm}{2cm}{0pt}{0pt}{0pt}{0pt}
\begin{document}
\maketitle
\section{The intrisic informations concerning the graph of extreme rays of 33BRO6.ext}
The graph considered is made of $21$ elements.\\
The group acting on 33BRO6.ext is Group3hmet6\\
Under the group action we have in fact only $2$ orbits to consider.\\
The diameter of the graph is $1$.\\
The list of orbits is with adjacency and Size info:
\begin{center}
\scriptsize
\begin{tabular}{cccccccccccccccc|c|c}
E1:&0&1&1&1&1&0&1&1&1&1&1&1&1&1&0&20&15\\
E2:&0&0&1&0&1&1&0&1&1&1&0&1&1&1&1&20&6\\
\end{tabular}
\end{center}
The representation matrix is:
\begin{center}
\scriptsize
\begin{tabular}{|c|cc|c|c|}
\hline
&E1&E2&Adj.&Size\\
\hline
E1& 14& 6&20&15\\
E2& 15& 5&20&6\\
\hline
Size&15&6&&21\\
\hline
\end{tabular}
\end{center}
\subsection{The local graphs}
The complemented local graph are:\\
We consider graph with cardinal inferior or equal to $20$\\
The complemented local graph for orbit $1$ is:
\begin{equation*}
\begin{array}{rrcl}
1&0&:&\\
2&0&:&\\
3&0&:&\\
4&0&:&\\
5&0&:&\\
6&0&:&\\
7&0&:&\\
8&0&:&\\
9&0&:&\\
10&0&:&\\
11&0&:&\\
12&0&:&\\
13&0&:&\\
14&0&:&\\
15&0&:&\\
16&0&:&\\
17&0&:&\\
18&0&:&\\
19&0&:&\\
20&0&:&\\
\end{array}
\end{equation*}
The complemented local graph for orbit $2$ is:
\begin{equation*}
\begin{array}{rrcl}
1&0&:&\\
2&0&:&\\
3&0&:&\\
4&0&:&\\
5&0&:&\\
6&0&:&\\
7&0&:&\\
8&0&:&\\
9&0&:&\\
10&0&:&\\
11&0&:&\\
12&0&:&\\
13&0&:&\\
14&0&:&\\
15&0&:&\\
16&0&:&\\
17&0&:&\\
18&0&:&\\
19&0&:&\\
20&0&:&\\
\end{array}
\end{equation*}
\subsection{scalar product between elements of orbits}
\noindent This subsection made sense only if the representation is orthogonal which is the usual case
Considering Orbit 1
\begin{enumerate}
\item Orbit Size is $15$.
\item Norm of elements is $12$.
\item Possible scalar product : $9$, $10$
\end{enumerate}
Considering Orbit 2
\begin{enumerate}
\item Orbit Size is $6$.
\item Norm of elements is $10$.
\item Possible scalar product : $6$
\end{enumerate}
\subsection{invariant group of Orbits}
\noindent The invariant group of $x$ is the set of $g$ such that g(x)=x denoted b $G_x$.\\
If $x$ and $y$ are in the same orbit then the groups $G_x$ and  $G_x$ are conjuguate.\\
Invariant group of Orbit 1
\begin{enumerate}
\item Group Size is $48$.
\item Elements are : (6 5 4 3 2 1  ), (6 5 4 3 1 2  ), (6 5 3 4 2 1  ), (6 5 3 4 1 2  ), (6 5 2 1 4 3  ), (6 5 1 2 4 3  ), (6 5 2 1 3 4  ), (6 5 1 2 3 4  ), (5 6 4 3 2 1  ), (5 6 4 3 1 2  ), (5 6 3 4 2 1  ), (5 6 3 4 1 2  ), (5 6 2 1 4 3  ), (5 6 1 2 4 3  ), (5 6 2 1 3 4  ), (5 6 1 2 3 4  ), (4 3 6 5 2 1  ), (4 3 6 5 1 2  ), (3 4 6 5 2 1  ), (3 4 6 5 1 2  ), (2 1 6 5 4 3  ), (1 2 6 5 4 3  ), (2 1 6 5 3 4  ), (1 2 6 5 3 4  ), (4 3 5 6 2 1  ), (4 3 5 6 1 2  ), (3 4 5 6 2 1  ), (3 4 5 6 1 2  ), (2 1 5 6 4 3  ), (1 2 5 6 4 3  ), (2 1 5 6 3 4  ), (1 2 5 6 3 4  ), (4 3 2 1 6 5  ), (4 3 1 2 6 5  ), (3 4 2 1 6 5  ), (3 4 1 2 6 5  ), (2 1 4 3 6 5  ), (1 2 4 3 6 5  ), (2 1 3 4 6 5  ), (1 2 3 4 6 5  ), (4 3 2 1 5 6  ), (4 3 1 2 5 6  ), (3 4 2 1 5 6  ), (3 4 1 2 5 6  ), (2 1 4 3 5 6  ), (1 2 4 3 5 6  ), (2 1 3 4 5 6  ), (1 2 3 4 5 6  ), 
\end{enumerate}
Invariant group of Orbit 2
\begin{enumerate}
\item Group Size is $120$.
\item Elements are : (5 4 3 2 1 6  ), (5 4 3 1 2 6  ), (5 4 2 3 1 6  ), (5 4 1 3 2 6  ), (5 4 2 1 3 6  ), (5 4 1 2 3 6  ), (5 3 4 2 1 6  ), (5 3 4 1 2 6  ), (5 2 4 3 1 6  ), (5 1 4 3 2 6  ), (5 2 4 1 3 6  ), (5 1 4 2 3 6  ), (5 3 2 4 1 6  ), (5 3 1 4 2 6  ), (5 2 3 4 1 6  ), (5 1 3 4 2 6  ), (5 2 1 4 3 6  ), (5 1 2 4 3 6  ), (5 3 2 1 4 6  ), (5 3 1 2 4 6  ), (5 2 3 1 4 6  ), (5 1 3 2 4 6  ), (5 2 1 3 4 6  ), (5 1 2 3 4 6  ), (4 5 3 2 1 6  ), (4 5 3 1 2 6  ), (4 5 2 3 1 6  ), (4 5 1 3 2 6  ), (4 5 2 1 3 6  ), (4 5 1 2 3 6  ), (3 5 4 2 1 6  ), (3 5 4 1 2 6  ), (2 5 4 3 1 6  ), (1 5 4 3 2 6  ), (2 5 4 1 3 6  ), (1 5 4 2 3 6  ), (3 5 2 4 1 6  ), (3 5 1 4 2 6  ), (2 5 3 4 1 6  ), (1 5 3 4 2 6  ), (2 5 1 4 3 6  ), (1 5 2 4 3 6  ), (3 5 2 1 4 6  ), (3 5 1 2 4 6  ), (2 5 3 1 4 6  ), (1 5 3 2 4 6  ), (2 5 1 3 4 6  ), (1 5 2 3 4 6  ), (4 3 5 2 1 6  ), (4 3 5 1 2 6  ), (4 2 5 3 1 6  ), (4 1 5 3 2 6  ), (4 2 5 1 3 6  ), (4 1 5 2 3 6  ), (3 4 5 2 1 6  ), (3 4 5 1 2 6  ), (2 4 5 3 1 6  ), (1 4 5 3 2 6  ), (2 4 5 1 3 6  ), (1 4 5 2 3 6  ), (3 2 5 4 1 6  ), (3 1 5 4 2 6  ), (2 3 5 4 1 6  ), (1 3 5 4 2 6  ), (2 1 5 4 3 6  ), (1 2 5 4 3 6  ), (3 2 5 1 4 6  ), (3 1 5 2 4 6  ), (2 3 5 1 4 6  ), (1 3 5 2 4 6  ), (2 1 5 3 4 6  ), (1 2 5 3 4 6  ), (4 3 2 5 1 6  ), (4 3 1 5 2 6  ), (4 2 3 5 1 6  ), (4 1 3 5 2 6  ), (4 2 1 5 3 6  ), (4 1 2 5 3 6  ), (3 4 2 5 1 6  ), (3 4 1 5 2 6  ), (2 4 3 5 1 6  ), (1 4 3 5 2 6  ), (2 4 1 5 3 6  ), (1 4 2 5 3 6  ), (3 2 4 5 1 6  ), (3 1 4 5 2 6  ), (2 3 4 5 1 6  ), (1 3 4 5 2 6  ), (2 1 4 5 3 6  ), (1 2 4 5 3 6  ), (3 2 1 5 4 6  ), (3 1 2 5 4 6  ), (2 3 1 5 4 6  ), (1 3 2 5 4 6  ), (2 1 3 5 4 6  ), (1 2 3 5 4 6  ), (4 3 2 1 5 6  ), (4 3 1 2 5 6  ), (4 2 3 1 5 6  ), (4 1 3 2 5 6  ), (4 2 1 3 5 6  ), (4 1 2 3 5 6  ), (3 4 2 1 5 6  ), (3 4 1 2 5 6  ), (2 4 3 1 5 6  ), (1 4 3 2 5 6  ), (2 4 1 3 5 6  ), (1 4 2 3 5 6  ), (3 2 4 1 5 6  ), (3 1 4 2 5 6  ), (2 3 4 1 5 6  ), (1 3 4 2 5 6  ), (2 1 4 3 5 6  ), (1 2 4 3 5 6  ), (3 2 1 4 5 6  ), (3 1 2 4 5 6  ), (2 3 1 4 5 6  ), (1 3 2 4 5 6  ), (2 1 3 4 5 6  ), (1 2 3 4 5 6  ), 
\end{enumerate}
\section{The intrisic informations concerning the graph of facet rays of 33BRO6.ine}
The graph considered is made of $150$ elements.\\
The group acting on 33BRO6.ine is Group3hmet6\\
Under the group action we have in fact only $3$ orbits to consider.\\
The diameter of the graph is $3$.\\
The list of orbits is with adjacency and Size info:
\begin{center}
\scriptsize
\begin{tabular}{cccccccccccccccc|c|c}
F1:&-1&-1&0&-1&0&0&0&1&1&1&0&1&1&1&-2&25&60\\
F2:&-3&0&1&0&1&0&0&1&0&0&0&1&0&0&0&25&30\\
F3:&-1&-1&1&-1&1&1&0&0&0&0&0&0&0&0&1&14&60\\
\end{tabular}
\end{center}
The representation matrix is:
\begin{center}
\scriptsize
\begin{tabular}{|c|ccc|c|c|}
\hline
&F1&F2&F3&Adj.&Size\\
\hline
F1& 12& 7& 6&25&60\\
F2& 14& 5& 6&25&30\\
F3& 6& 3& 5&14&60\\
\hline
Size&60&30&60&&150\\
\hline
\end{tabular}
\end{center}
\subsection{The local graphs}
The complemented local graph are:\\
We consider graph with cardinal inferior or equal to $20$\\
The complemented local graph for orbit $3$ is:
\begin{equation*}
\begin{array}{rrcl}
1&9&:&\,\,2\,\,5\,\,6\,\,7\,\,8\,\,9\,\,10\,\,11\,\,12\\
2&5&:&\,\,1\,\,3\,\,4\,\,13\,\,14\\
3&9&:&\,\,2\,\,5\,\,6\,\,7\,\,8\,\,9\,\,10\,\,11\,\,12\\
4&9&:&\,\,2\,\,5\,\,6\,\,7\,\,8\,\,9\,\,10\,\,11\,\,12\\
5&5&:&\,\,1\,\,3\,\,4\,\,13\,\,14\\
6&5&:&\,\,1\,\,3\,\,4\,\,13\,\,14\\
7&5&:&\,\,1\,\,3\,\,4\,\,13\,\,14\\
8&5&:&\,\,1\,\,3\,\,4\,\,13\,\,14\\
9&5&:&\,\,1\,\,3\,\,4\,\,13\,\,14\\
10&5&:&\,\,1\,\,3\,\,4\,\,13\,\,14\\
11&5&:&\,\,1\,\,3\,\,4\,\,13\,\,14\\
12&5&:&\,\,1\,\,3\,\,4\,\,13\,\,14\\
13&9&:&\,\,2\,\,5\,\,6\,\,7\,\,8\,\,9\,\,10\,\,11\,\,12\\
14&9&:&\,\,2\,\,5\,\,6\,\,7\,\,8\,\,9\,\,10\,\,11\,\,12\\
\end{array}
\end{equation*}
\subsection{scalar product between elements of orbits}
\noindent This subsection made sense only if the representation is orthogonal which is the usual case
Considering Orbit 1
\begin{enumerate}
\item Orbit Size is $60$.
\item Norm of elements is $13$.
\item Possible scalar product : $-3$, $1$, $5$, $9$
\end{enumerate}
Considering Orbit 2
\begin{enumerate}
\item Orbit Size is $30$.
\item Norm of elements is $13$.
\item Possible scalar product : $-3$, $1$, $9$
\end{enumerate}
Considering Orbit 3
\begin{enumerate}
\item Orbit Size is $60$.
\item Norm of elements is $7$.
\item Possible scalar product : $-1$, $3$
\end{enumerate}
\subsection{invariant group of Orbits}
\noindent The invariant group of $x$ is the set of $g$ such that g(x)=x denoted b $G_x$.\\
If $x$ and $y$ are in the same orbit then the groups $G_x$ and  $G_x$ are conjuguate.\\
Invariant group of Orbit 1
\begin{enumerate}
\item Group Size is $12$.
\item Elements are : (2 1 5 4 3 6  ), (1 2 5 4 3 6  ), (2 1 5 3 4 6  ), (1 2 5 3 4 6  ), (2 1 4 5 3 6  ), (1 2 4 5 3 6  ), (2 1 3 5 4 6  ), (1 2 3 5 4 6  ), (2 1 4 3 5 6  ), (1 2 4 3 5 6  ), (2 1 3 4 5 6  ), (1 2 3 4 5 6  ), 
\end{enumerate}
Invariant group of Orbit 2
\begin{enumerate}
\item Group Size is $24$.
\item Elements are : (4 3 2 1 5 6  ), (4 3 1 2 5 6  ), (4 2 3 1 5 6  ), (4 1 3 2 5 6  ), (4 2 1 3 5 6  ), (4 1 2 3 5 6  ), (3 4 2 1 5 6  ), (3 4 1 2 5 6  ), (2 4 3 1 5 6  ), (1 4 3 2 5 6  ), (2 4 1 3 5 6  ), (1 4 2 3 5 6  ), (3 2 4 1 5 6  ), (3 1 4 2 5 6  ), (2 3 4 1 5 6  ), (1 3 4 2 5 6  ), (2 1 4 3 5 6  ), (1 2 4 3 5 6  ), (3 2 1 4 5 6  ), (3 1 2 4 5 6  ), (2 3 1 4 5 6  ), (1 3 2 4 5 6  ), (2 1 3 4 5 6  ), (1 2 3 4 5 6  ), 
\end{enumerate}
Invariant group of Orbit 3
\begin{enumerate}
\item Group Size is $12$.
\item Elements are : (2 1 5 4 3 6  ), (1 2 5 4 3 6  ), (2 1 5 3 4 6  ), (1 2 5 3 4 6  ), (2 1 4 5 3 6  ), (1 2 4 5 3 6  ), (2 1 3 5 4 6  ), (1 2 3 5 4 6  ), (2 1 4 3 5 6  ), (1 2 4 3 5 6  ), (2 1 3 4 5 6  ), (1 2 3 4 5 6  ), 
\end{enumerate}
\section{The two incidence matrix}
Incidence between Orbit of Extreme rays and Orbits of Facets
\begin{equation*}
\begin{array}{|c|ccc|c|}
\hline
&F1&F2&F3&Inc\\
\hline
E1&48&24&36&108\\
E2&50&25&50&125\\
\hline
\end{array}
\end{equation*}
Incidence between Orbits of Facets and Orbits of Extreme Rays
\begin{equation*}
\begin{array}{|c|cc|c|}
\hline
&E1&E2&Inc\\
\hline
F1&12&5&17\\
F2&12&5&17\\
F3&9&5&14\\
\hline
\end{array}
\end{equation*}
\end{document}
