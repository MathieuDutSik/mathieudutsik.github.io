\documentclass[12pt]{article}
\usepackage{graphicx,amssymb,amsmath,vmargin,multicol}
\usepackage[english]{babel}
\title{The Cone 2-P5}
\setpapersize{custom}{21cm}{29.7cm}
\setmarginsrb{1cm}{1cm}{1cm}{2cm}{0pt}{0pt}{0pt}{0pt}
\begin{document}
\maketitle
\section{The intrisic informations concerning the graph of extreme rays of 2-P5.ext}
The graph considered is made of $25$ elements.\\
The group acting on 2-P5.ext is Group2hmet5\\
Under the group action we have in fact only $2$ orbits to consider.\\
The diameter of the graph is $2$.\\
The list of orbits is with adjacency and Size info:
\begin{center}
\scriptsize
\begin{tabular}{ccccccccccc|c|c}
E1:&0&0&0&0&1&1&0&1&1&0&22&15\\
E2:&0&0&0&0&0&1&0&0&1&1&21&10\\
\end{tabular}
\end{center}
The representation matrix is:
\begin{center}
\scriptsize
\begin{tabular}{|c|cc|c|c|}
\hline
&E1&E2&Adj.&Size\\
\hline
E1& 14& 8&22&15\\
E2& 12& 9&21&10\\
\hline
Size&15&10&&25\\
\hline
\end{tabular}
\end{center}
No complemented local graph of size inferior to $20$.
\subsection{scalar product between elements of orbits}
\noindent This subsection made sense only if the representation is orthogonal which is the usual case
Considering Orbit 1
\begin{enumerate}
\item Orbit Size is $15$.
\item Norm of elements is $4$.
\item Possible scalar product : $1$, $2$
\end{enumerate}
Considering Orbit 2
\begin{enumerate}
\item Orbit Size is $10$.
\item Norm of elements is $3$.
\item Possible scalar product : $0$, $1$
\end{enumerate}
\subsection{invariant group of Orbits}
\noindent The invariant group of $x$ is the set of $g$ such that g(x)=x denoted b $G_x$.\\
If $x$ and $y$ are in the same orbit then the groups $G_x$ and  $G_x$ are conjuguate.\\
Invariant group of Orbit 1
\begin{enumerate}
\item Group Size is $8$.
\item Elements are : (4 3 2 1 5  ), (4 3 1 2 5  ), (3 4 2 1 5  ), (3 4 1 2 5  ), (2 1 4 3 5  ), (1 2 4 3 5  ), (2 1 3 4 5  ), (1 2 3 4 5  ), 
\end{enumerate}
Invariant group of Orbit 2
\begin{enumerate}
\item Group Size is $12$.
\item Elements are : (3 2 1 5 4  ), (3 1 2 5 4  ), (2 3 1 5 4  ), (1 3 2 5 4  ), (2 1 3 5 4  ), (1 2 3 5 4  ), (3 2 1 4 5  ), (3 1 2 4 5  ), (2 3 1 4 5  ), (1 3 2 4 5  ), (2 1 3 4 5  ), (1 2 3 4 5  ), 
\end{enumerate}
\section{The intrisic informations concerning the graph of facet rays of 2-P5.ine}
The graph considered is made of $120$ elements.\\
The group acting on 2-P5.ine is Group2hmet5\\
Under the group action we have in fact only $4$ orbits to consider.\\
The diameter of the graph is $3$.\\
The list of orbits is with adjacency and Size info:
\begin{center}
\scriptsize
\begin{tabular}{ccccccccccc|c|c}
F1:&-1&0&1&0&1&0&0&1&0&0&49&20\\
F2:&0&0&0&0&0&0&0&0&0&1&42&10\\
F3:&-1&0&1&1&0&1&1&2&-1&0&15&60\\
F4:&-1&-1&2&2&1&1&2&1&1&-2&9&30\\
\end{tabular}
\end{center}
The representation matrix is:
\begin{center}
\scriptsize
\begin{tabular}{|c|cccc|c|c|}
\hline
&F1&F2&F3&F4&Adj.&Size\\
\hline
F1& 16& 9& 18& 6&49&20\\
F2& 18& 3& 18& 3&42&10\\
F3& 6& 3& 4& 2&15&60\\
F4& 4& 1& 4& 0&9&30\\
\hline
Size&20&10&60&30&&120\\
\hline
\end{tabular}
\end{center}
\subsection{The local graphs}
The complemented local graph are:\\
We consider graph with cardinal inferior or equal to $20$\\
The complemented local graph for orbit $3$ is:
\begin{equation*}
\begin{array}{rrcl}
1&8&:&\,\,4\,\,5\,\,6\,\,7\,\,8\,\,9\,\,10\,\,14\\
2&5&:&\,\,4\,\,5\,\,6\,\,10\,\,14\\
3&2&:&\,\,7\,\,8\\
4&8&:&\,\,1\,\,2\,\,7\,\,8\,\,10\,\,11\,\,13\,\,15\\
5&4&:&\,\,1\,\,2\,\,7\,\,13\\
6&4&:&\,\,1\,\,2\,\,8\,\,11\\
7&7&:&\,\,1\,\,3\,\,4\,\,5\,\,11\,\,12\,\,14\\
8&7&:&\,\,1\,\,3\,\,4\,\,6\,\,12\,\,13\,\,15\\
9&5&:&\,\,1\,\,10\,\,11\,\,13\,\,15\\
10&4&:&\,\,1\,\,2\,\,4\,\,9\\
11&4&:&\,\,4\,\,6\,\,7\,\,9\\
12&2&:&\,\,7\,\,8\\
13&4&:&\,\,4\,\,5\,\,8\,\,9\\
14&3&:&\,\,1\,\,2\,\,7\\
15&3&:&\,\,4\,\,8\,\,9\\
\end{array}
\end{equation*}
The complemented local graph for orbit $4$ is:
\begin{equation*}
\begin{array}{rrcl}
1&0&:&\\
2&0&:&\\
3&2&:&\,\,4\,\,6\\
4&2&:&\,\,3\,\,5\\
5&2&:&\,\,4\,\,6\\
6&2&:&\,\,3\,\,5\\
7&0&:&\\
8&0&:&\\
9&0&:&\\
\end{array}
\end{equation*}
\subsection{scalar product between elements of orbits}
\noindent This subsection made sense only if the representation is orthogonal which is the usual case
Considering Orbit 1
\begin{enumerate}
\item Orbit Size is $20$.
\item Norm of elements is $4$.
\item Possible scalar product : $-1$, $0$, $1$
\end{enumerate}
Considering Orbit 2
\begin{enumerate}
\item Orbit Size is $10$.
\item Norm of elements is $1$.
\item Possible scalar product : $0$
\end{enumerate}
Considering Orbit 3
\begin{enumerate}
\item Orbit Size is $60$.
\item Norm of elements is $10$.
\item Possible scalar product : $-5$, $-4$, $-2$, $-1$, $0$, $2$, $3$, $4$, $7$, $8$
\end{enumerate}
Considering Orbit 4
\begin{enumerate}
\item Orbit Size is $30$.
\item Norm of elements is $22$.
\item Possible scalar product : $-12$, $-5$, $-2$, $3$, $4$, $8$, $11$, $15$
\end{enumerate}
\subsection{invariant group of Orbits}
\noindent The invariant group of $x$ is the set of $g$ such that g(x)=x denoted b $G_x$.\\
If $x$ and $y$ are in the same orbit then the groups $G_x$ and  $G_x$ are conjuguate.\\
Invariant group of Orbit 1
\begin{enumerate}
\item Group Size is $6$.
\item Elements are : (3 2 1 4 5  ), (3 1 2 4 5  ), (2 3 1 4 5  ), (1 3 2 4 5  ), (2 1 3 4 5  ), (1 2 3 4 5  ), 
\end{enumerate}
Invariant group of Orbit 2
\begin{enumerate}
\item Group Size is $12$.
\item Elements are : (2 1 5 4 3  ), (1 2 5 4 3  ), (2 1 5 3 4  ), (1 2 5 3 4  ), (2 1 4 5 3  ), (1 2 4 5 3  ), (2 1 3 5 4  ), (1 2 3 5 4  ), (2 1 4 3 5  ), (1 2 4 3 5  ), (2 1 3 4 5  ), (1 2 3 4 5  ), 
\end{enumerate}
Invariant group of Orbit 3
\begin{enumerate}
\item Group Size is $2$.
\item Elements are : (4 2 5 1 3  ), (1 2 3 4 5  ), 
\end{enumerate}
Invariant group of Orbit 4
\begin{enumerate}
\item Group Size is $4$.
\item Elements are : (2 1 4 3 5  ), (1 2 4 3 5  ), (2 1 3 4 5  ), (1 2 3 4 5  ), 
\end{enumerate}
\section{The two incidence matrix}
Incidence between Orbit of Extreme rays and Orbits of Facets
\begin{equation*}
\begin{array}{|c|cccc|c|}
\hline
&F1&F2&F3&F4&Inc\\
\hline
E1&12&6&24&12&54\\
E2&14&7&24&9&54\\
\hline
\end{array}
\end{equation*}
Incidence between Orbits of Facets and Orbits of Extreme Rays
\begin{equation*}
\begin{array}{|c|cc|c|}
\hline
&E1&E2&Inc\\
\hline
F1&9&7&16\\
F2&9&7&16\\
F3&6&4&10\\
F4&6&3&9\\
\hline
\end{array}
\end{equation*}
\end{document}
