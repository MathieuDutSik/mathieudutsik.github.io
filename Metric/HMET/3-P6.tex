\documentclass[12pt]{article}
\usepackage{graphicx,amssymb,amsmath,vmargin,multicol}
\usepackage[english]{babel}
\title{The Cone 3-P6}
\setpapersize{custom}{21cm}{29.7cm}
\setmarginsrb{1cm}{1cm}{1cm}{2cm}{0pt}{0pt}{0pt}{0pt}
\begin{document}
\maketitle
\section{The intrisic informations concerning the graph of extreme rays of 3-P6.ext}
The graph considered is made of $65$ elements.\\
The group acting on 3-P6.ext is ReprS6\\
Under the group action we have in fact only $2$ orbits to consider.\\
The diameter of the graph is $2$.\\
The list of orbits is with adjacency and Size info:
\begin{center}
\scriptsize
\begin{tabular}{cccccccccccccccc|c|c}
E1:&0&0&0&0&0&0&0&0&1&1&0&0&1&1&0&58&45\\
E2:&0&0&0&0&0&0&0&0&0&1&0&0&0&1&1&55&20\\
\end{tabular}
\end{center}
The representation matrix is:
\begin{center}
\scriptsize
\begin{tabular}{|c|cc|c|c|}
\hline
&E1&E2&Adj.&Size\\
\hline
E1& 42& 16&58&45\\
E2& 36& 19&55&20\\
\hline
Size&45&20&&65\\
\hline
\end{tabular}
\end{center}
No complemented local graph of size inferior to $20$.
\subsection{scalar product between elements of orbits}
\noindent This subsection made sense only if the representation is orthogonal which is the usual case
Considering Orbit 1
\begin{enumerate}
\item Orbit Size is $45$.
\item Norm of elements is $4$.
\item Possible scalar product : $0$, $1$, $2$
\end{enumerate}
Considering Orbit 2
\begin{enumerate}
\item Orbit Size is $20$.
\item Norm of elements is $3$.
\item Possible scalar product : $0$, $1$
\end{enumerate}
\subsection{invariant group of Orbits}
\noindent The invariant group of $x$ is the set of $g$ such that $g(x)=x$ denoted by $G_x$.\\
If $x$ and $y$ are in the same orbit then the groups $G_x$ and  $G_y$ are conjuguate.\\
For any element $x$ we denote $O_x$ its Orbit, we have $|O_x|\times |G_x|=|G|$
We print only nontrivial groups
Invariant group of Orbit $1$:
\begin{enumerate}
\item Group Size is $16$.
\item Elements are : (4 3 2 1 6 5  ), (4 3 1 2 6 5  ), (3 4 2 1 6 5  ), (3 4 1 2 6 5  ), (2 1 4 3 6 5  ), (1 2 4 3 6 5  ), (2 1 3 4 6 5  ), (1 2 3 4 6 5  ), (4 3 2 1 5 6  ), (4 3 1 2 5 6  ), (3 4 2 1 5 6  ), (3 4 1 2 5 6  ), (2 1 4 3 5 6  ), (1 2 4 3 5 6  ), (2 1 3 4 5 6  ), (1 2 3 4 5 6  )
\end{enumerate}
Invariant group of Orbit $2$:
\begin{enumerate}
\item Group Size is $36$.
\item Elements are : (3 2 1 6 5 4  ), (3 1 2 6 5 4  ), (2 3 1 6 5 4  ), (1 3 2 6 5 4  ), (2 1 3 6 5 4  ), (1 2 3 6 5 4  ), (3 2 1 6 4 5  ), (3 1 2 6 4 5  ), (2 3 1 6 4 5  ), (1 3 2 6 4 5  ), (2 1 3 6 4 5  ), (1 2 3 6 4 5  ), (3 2 1 5 6 4  ), (3 1 2 5 6 4  ), (2 3 1 5 6 4  ), (1 3 2 5 6 4  ), (2 1 3 5 6 4  ), (1 2 3 5 6 4  ), (3 2 1 4 6 5  ), (3 1 2 4 6 5  ), (2 3 1 4 6 5  ), (1 3 2 4 6 5  ), (2 1 3 4 6 5  ), (1 2 3 4 6 5  ), (3 2 1 5 4 6  ), (3 1 2 5 4 6  ), (2 3 1 5 4 6  ), (1 3 2 5 4 6  ), (2 1 3 5 4 6  ), (1 2 3 5 4 6  ), (3 2 1 4 5 6  ), (3 1 2 4 5 6  ), (2 3 1 4 5 6  ), (1 3 2 4 5 6  ), (2 1 3 4 5 6  ), (1 2 3 4 5 6  )
\end{enumerate}
\section{The intrisic informations concerning the graph of facet rays of 3-P6.ine}
The graph considered is made of $4065$ elements.\\
The group acting on 3-P6.ine is ReprS6\\
Under the group action we have in fact only $16$ orbits to consider.\\
The diameter of the graph is $3$.\\
The list of orbits is with adjacency and Size info:
\begin{center}
\scriptsize
\begin{tabular}{cccccccccccccccc|c|c}
F1:&0&0&0&0&0&0&0&0&0&0&0&0&0&0&1&1526&15\\
F2:&-1&0&1&0&1&0&0&1&0&0&0&1&0&0&0&703&30\\
F3:&-1&0&1&0&1&0&1&0&1&1&2&1&0&0&-1&100&180\\
F4:&-1&0&1&0&1&0&1&0&1&1&1&2&-1&1&0&37&360\\
F5:&-1&-1&2&0&1&1&2&1&1&0&2&1&1&2&-2&31&360\\
F6:&-1&-1&2&1&0&2&1&2&0&2&2&1&1&-1&-1&30&180\\
F7:&-1&-1&2&1&0&2&2&1&1&-1&2&3&-1&1&0&23&360\\
F8:&-1&-2&3&1&2&1&2&1&2&-1&2&1&2&3&-2&23&360\\
F9:&-1&-2&3&2&1&2&2&1&2&-2&3&4&-1&1&1&23&180\\
F10:&-1&-1&2&-1&2&2&2&1&1&1&2&1&1&1&-2&22&60\\
F11:&-1&-1&2&0&1&1&1&2&2&-1&2&1&1&2&-1&18&360\\
F12:&-1&-1&2&1&1&1&1&1&1&-1&1&1&1&-1&2&18&90\\
F13:&-1&-1&2&1&0&2&1&2&0&0&2&1&1&-1&1&14&720\\
F14:&-1&0&1&0&1&2&2&1&0&2&2&1&2&0&-2&14&360\\
F15:&-1&-1&2&1&0&2&2&1&1&1&2&1&1&1&-2&14&360\\
F16:&-1&0&1&0&1&0&1&0&1&1&1&0&1&1&0&14&90\\
\end{tabular}
\end{center}
The representation matrix is:
\begin{center}
\scriptsize
\begin{tabular}{|c|cccccccccccccccc|c|c|}
\hline
&F1&F2&F3&F4&F5&F6&F7&F8&F9&F10&F11&F12&F13&F14&F15&F16&Adj.&Size\\
\hline
F1& 14& 28& 144& 144& 192& 96& 120& 168& 36& 20& 120& 36& 240& 72& 72& 24&1526&15\\
F2& 14& 25& 72& 96& 60& 24& 72& 48& 36& 10& 48& 12& 72& 48& 48& 18&703&30\\
F3& 12& 12& 14& 10& 10& 6& 10& 0& 0& 2& 2& 0& 12& 4& 4& 2&100&180\\
F4& 6& 8& 5& 2& 2& 2& 2& 2& 2& 0& 2& 0& 0& 2& 2& 0&37&360\\
F5& 8& 5& 5& 2& 3& 0& 0& 2& 0& 1& 1& 0& 0& 2& 2& 0&31&360\\
F6& 8& 4& 6& 4& 0& 2& 4& 2& 0& 0& 0& 0& 0& 0& 0& 0&30&180\\
F7& 5& 6& 5& 2& 0& 2& 0& 0& 1& 0& 0& 0& 2& 0& 0& 0&23&360\\
F8& 7& 4& 0& 2& 2& 1& 0& 0& 2& 0& 2& 1& 2& 0& 0& 0&23&360\\
F9& 3& 6& 0& 4& 0& 0& 2& 4& 0& 0& 0& 0& 0& 2& 2& 0&23&180\\
F10& 5& 5& 6& 0& 6& 0& 0& 0& 0& 0& 0& 0& 0& 0& 0& 0&22&60\\
F11& 5& 4& 1& 2& 1& 0& 0& 2& 0& 0& 0& 1& 2& 0& 0& 0&18&360\\
F12& 6& 4& 0& 0& 0& 0& 0& 4& 0& 0& 4& 0& 0& 0& 0& 0&18&90\\
F13& 5& 3& 3& 0& 0& 0& 1& 1& 0& 0& 1& 0& 0& 0& 0& 0&14&720\\
F14& 3& 4& 2& 2& 2& 0& 0& 0& 1& 0& 0& 0& 0& 0& 0& 0&14&360\\
F15& 3& 4& 2& 2& 2& 0& 0& 0& 1& 0& 0& 0& 0& 0& 0& 0&14&360\\
F16& 4& 6& 4& 0& 0& 0& 0& 0& 0& 0& 0& 0& 0& 0& 0& 0&14&90\\
\hline
Size&15&30&180&360&360&180&360&360&180&60&360&90&720&360&360&90&&4065\\
\hline
\end{tabular}
\end{center}
\subsection{The local graphs}
The complemented local graph are:\\
We consider graph with cardinal inferior or equal to $20$\\
The complemented local graph for orbit $11$ is:
\begin{equation*}
\begin{array}{rrcl}
1&0&:&\\
2&8&:&\,\,3\,\,4\,\,6\,\,7\,\,10\,\,12\,\,13\,\,17\\
3&8&:&\,\,2\,\,4\,\,5\,\,7\,\,11\,\,12\,\,13\,\,15\\
4&5&:&\,\,2\,\,3\,\,7\,\,12\,\,13\\
5&5&:&\,\,3\,\,6\,\,9\,\,12\,\,13\\
6&5&:&\,\,2\,\,5\,\,9\,\,12\,\,13\\
7&6&:&\,\,2\,\,3\,\,4\,\,9\,\,12\,\,13\\
8&0&:&\\
9&3&:&\,\,5\,\,6\,\,7\\
10&1&:&\,\,2\\
11&1&:&\,\,3\\
12&7&:&\,\,2\,\,3\,\,4\,\,5\,\,6\,\,7\,\,13\\
13&7&:&\,\,2\,\,3\,\,4\,\,5\,\,6\,\,7\,\,12\\
14&0&:&\\
15&1&:&\,\,3\\
16&0&:&\\
17&1&:&\,\,2\\
18&0&:&\\
\end{array}
\end{equation*}
The complemented local graph for orbit $12$ is:
\begin{equation*}
\begin{array}{rrcl}
1&0&:&\\
2&6&:&\,\,3\,\,4\,\,5\,\,6\,\,9\,\,16\\
3&6&:&\,\,2\,\,4\,\,5\,\,7\,\,8\,\,15\\
4&6&:&\,\,2\,\,3\,\,5\,\,8\,\,9\,\,10\\
5&6&:&\,\,2\,\,3\,\,4\,\,6\,\,7\,\,11\\
6&5&:&\,\,2\,\,5\,\,7\,\,8\,\,9\\
7&5&:&\,\,3\,\,5\,\,6\,\,8\,\,9\\
8&5&:&\,\,3\,\,4\,\,6\,\,7\,\,9\\
9&5&:&\,\,2\,\,4\,\,6\,\,7\,\,8\\
10&1&:&\,\,4\\
11&1&:&\,\,5\\
12&0&:&\\
13&0&:&\\
14&0&:&\\
15&1&:&\,\,3\\
16&1&:&\,\,2\\
17&0&:&\\
18&0&:&\\
\end{array}
\end{equation*}
The complemented local graph for orbit $13$ is:
\begin{equation*}
\begin{array}{rrcl}
1&0&:&\\
2&4&:&\,\,3\,\,7\,\,8\,\,10\\
3&3&:&\,\,2\,\,4\,\,14\\
4&4&:&\,\,3\,\,6\,\,8\,\,10\\
5&0&:&\\
6&1&:&\,\,4\\
7&1&:&\,\,2\\
8&2&:&\,\,2\,\,4\\
9&0&:&\\
10&2&:&\,\,2\,\,4\\
11&0&:&\\
12&0&:&\\
13&0&:&\\
14&1&:&\,\,3\\
\end{array}
\end{equation*}
The complemented local graph for orbit $14$ is:
\begin{equation*}
\begin{array}{rrcl}
1&1&:&\,\,3\\
2&4&:&\,\,3\,\,6\,\,7\,\,13\\
3&3&:&\,\,1\,\,2\,\,8\\
4&1&:&\,\,6\\
5&0&:&\\
6&3&:&\,\,2\,\,4\,\,10\\
7&2&:&\,\,2\,\,10\\
8&3&:&\,\,3\,\,10\,\,13\\
9&0&:&\\
10&3&:&\,\,6\,\,7\,\,8\\
11&0&:&\\
12&0&:&\\
13&2&:&\,\,2\,\,8\\
14&0&:&\\
\end{array}
\end{equation*}
The complemented local graph for orbit $15$ is:
\begin{equation*}
\begin{array}{rrcl}
1&0&:&\\
2&4&:&\,\,3\,\,4\,\,13\,\,14\\
3&3&:&\,\,2\,\,8\,\,10\\
4&3&:&\,\,2\,\,7\,\,10\\
5&0&:&\\
6&0&:&\\
7&3&:&\,\,4\,\,8\,\,13\\
8&3&:&\,\,3\,\,7\,\,14\\
9&0&:&\\
10&2&:&\,\,3\,\,4\\
11&0&:&\\
12&0&:&\\
13&2&:&\,\,2\,\,7\\
14&2&:&\,\,2\,\,8\\
\end{array}
\end{equation*}
The complemented local graph for orbit $16$ is:
\begin{equation*}
\begin{array}{rrcl}
1&1&:&\,\,7\\
2&1&:&\,\,3\\
3&1&:&\,\,2\\
4&0&:&\\
5&0&:&\\
6&0&:&\\
7&1&:&\,\,1\\
8&1&:&\,\,9\\
9&1&:&\,\,8\\
10&1&:&\,\,11\\
11&1&:&\,\,10\\
12&0&:&\\
13&0&:&\\
14&0&:&\\
\end{array}
\end{equation*}
\subsection{scalar product between elements of orbits}
\noindent This subsection made sense only if the representation is orthogonal which is the usual case
Considering Orbit 1
\begin{enumerate}
\item Orbit Size is $15$.
\item Norm of elements is $1$.
\item Possible scalar product : $0$
\end{enumerate}
Considering Orbit 2
\begin{enumerate}
\item Orbit Size is $30$.
\item Norm of elements is $5$.
\item Possible scalar product : $-1$, $1$
\end{enumerate}
Considering Orbit 3
\begin{enumerate}
\item Orbit Size is $180$.
\item Norm of elements is $12$.
\item Possible scalar product : $-4$, $0$, $4$, $8$
\end{enumerate}
Considering Orbit 4
\begin{enumerate}
\item Orbit Size is $360$.
\item Norm of elements is $13$.
\item Possible scalar product : $-3$, $-1$, $1$, $3$, $5$, $7$, $9$
\end{enumerate}
Considering Orbit 5
\begin{enumerate}
\item Orbit Size is $360$.
\item Norm of elements is $28$.
\item Possible scalar product : $-8$, $-6$, $-4$, $-2$, $0$, $2$, $4$, $6$, $8$, $10$, $12$, $14$, $16$, $18$, $20$, $24$
\end{enumerate}
Considering Orbit 6
\begin{enumerate}
\item Orbit Size is $180$.
\item Norm of elements is $28$.
\item Possible scalar product : $-8$, $-4$, $0$, $4$, $8$, $12$, $16$, $20$, $24$
\end{enumerate}
Considering Orbit 7
\begin{enumerate}
\item Orbit Size is $360$.
\item Norm of elements is $33$.
\item Possible scalar product : $-7$, $-5$, $-3$, $-1$, $1$, $3$, $5$, $7$, $9$, $11$, $13$, $15$, $17$, $19$, $21$, $23$, $29$
\end{enumerate}
Considering Orbit 8
\begin{enumerate}
\item Orbit Size is $360$.
\item Norm of elements is $52$.
\item Possible scalar product : $-12$, $-8$, $-4$, $0$, $4$, $8$, $12$, $16$, $20$, $24$, $28$, $32$, $36$, $48$
\end{enumerate}
Considering Orbit 9
\begin{enumerate}
\item Orbit Size is $180$.
\item Norm of elements is $64$.
\item Possible scalar product : $-8$, $0$, $8$, $12$, $16$, $20$, $24$, $28$, $32$, $36$, $44$, $48$
\end{enumerate}
Considering Orbit 10
\begin{enumerate}
\item Orbit Size is $60$.
\item Norm of elements is $33$.
\item Possible scalar product : $-3$, $-1$, $5$, $7$, $9$, $13$, $21$
\end{enumerate}
Considering Orbit 11
\begin{enumerate}
\item Orbit Size is $360$.
\item Norm of elements is $29$.
\item Possible scalar product : $-5$, $-3$, $-1$, $1$, $3$, $5$, $7$, $9$, $11$, $13$, $15$, $17$, $21$, $25$
\end{enumerate}
Considering Orbit 12
\begin{enumerate}
\item Orbit Size is $90$.
\item Norm of elements is $21$.
\item Possible scalar product : $-1$, $2$, $3$, $5$, $8$, $11$, $12$
\end{enumerate}
Considering Orbit 13
\begin{enumerate}
\item Orbit Size is $720$.
\item Norm of elements is $24$.
\item Possible scalar product : $-6$, $-4$, $-2$, $0$, $2$, $4$, $6$, $8$, $10$, $12$, $14$, $16$, $18$, $20$
\end{enumerate}
Considering Orbit 14
\begin{enumerate}
\item Orbit Size is $360$.
\item Norm of elements is $29$.
\item Possible scalar product : $-7$, $-5$, $-3$, $-1$, $1$, $3$, $5$, $7$, $9$, $11$, $13$, $15$, $17$, $19$, $21$, $25$
\end{enumerate}
Considering Orbit 15
\begin{enumerate}
\item Orbit Size is $360$.
\item Norm of elements is $29$.
\item Possible scalar product : $-7$, $-5$, $-3$, $-1$, $1$, $3$, $5$, $7$, $9$, $11$, $13$, $15$, $17$, $19$, $21$, $25$
\end{enumerate}
Considering Orbit 16
\begin{enumerate}
\item Orbit Size is $90$.
\item Norm of elements is $9$.
\item Possible scalar product : $1$, $3$, $5$
\end{enumerate}
\subsection{invariant group of Orbits}
\noindent The invariant group of $x$ is the set of $g$ such that $g(x)=x$ denoted by $G_x$.\\
If $x$ and $y$ are in the same orbit then the groups $G_x$ and  $G_y$ are conjuguate.\\
For any element $x$ we denote $O_x$ its Orbit, we have $|O_x|\times |G_x|=|G|$
We print only nontrivial groups
Invariant group of Orbit $1$:
\begin{enumerate}
\item Group Size is $48$.
\item Elements are : (2 1 6 5 4 3  ), (1 2 6 5 4 3  ), (2 1 6 5 3 4  ), (1 2 6 5 3 4  ), (2 1 6 4 5 3  ), (1 2 6 4 5 3  ), (2 1 6 3 5 4  ), (1 2 6 3 5 4  ), (2 1 6 4 3 5  ), (1 2 6 4 3 5  ), (2 1 6 3 4 5  ), (1 2 6 3 4 5  ), (2 1 5 6 4 3  ), (1 2 5 6 4 3  ), (2 1 5 6 3 4  ), (1 2 5 6 3 4  ), (2 1 4 6 5 3  ), (1 2 4 6 5 3  ), (2 1 3 6 5 4  ), (1 2 3 6 5 4  ), (2 1 4 6 3 5  ), (1 2 4 6 3 5  ), (2 1 3 6 4 5  ), (1 2 3 6 4 5  ), (2 1 5 4 6 3  ), (1 2 5 4 6 3  ), (2 1 5 3 6 4  ), (1 2 5 3 6 4  ), (2 1 4 5 6 3  ), (1 2 4 5 6 3  ), (2 1 3 5 6 4  ), (1 2 3 5 6 4  ), (2 1 4 3 6 5  ), (1 2 4 3 6 5  ), (2 1 3 4 6 5  ), (1 2 3 4 6 5  ), (2 1 5 4 3 6  ), (1 2 5 4 3 6  ), (2 1 5 3 4 6  ), (1 2 5 3 4 6  ), (2 1 4 5 3 6  ), (1 2 4 5 3 6  ), (2 1 3 5 4 6  ), (1 2 3 5 4 6  ), (2 1 4 3 5 6  ), (1 2 4 3 5 6  ), (2 1 3 4 5 6  ), (1 2 3 4 5 6  )
\end{enumerate}
Invariant group of Orbit $2$:
\begin{enumerate}
\item Group Size is $24$.
\item Elements are : (4 3 2 1 5 6  ), (4 3 1 2 5 6  ), (4 2 3 1 5 6  ), (4 1 3 2 5 6  ), (4 2 1 3 5 6  ), (4 1 2 3 5 6  ), (3 4 2 1 5 6  ), (3 4 1 2 5 6  ), (2 4 3 1 5 6  ), (1 4 3 2 5 6  ), (2 4 1 3 5 6  ), (1 4 2 3 5 6  ), (3 2 4 1 5 6  ), (3 1 4 2 5 6  ), (2 3 4 1 5 6  ), (1 3 4 2 5 6  ), (2 1 4 3 5 6  ), (1 2 4 3 5 6  ), (3 2 1 4 5 6  ), (3 1 2 4 5 6  ), (2 3 1 4 5 6  ), (1 3 2 4 5 6  ), (2 1 3 4 5 6  ), (1 2 3 4 5 6  )
\end{enumerate}
Invariant group of Orbit $3$:
\begin{enumerate}
\item Group Size is $4$.
\item Elements are : (6 5 4 3 2 1  ), (6 5 3 4 2 1  ), (1 2 4 3 5 6  ), (1 2 3 4 5 6  )
\end{enumerate}
Invariant group of Orbit $4$:
\begin{enumerate}
\item Group Size is $2$.
\item Elements are : (5 2 3 6 1 4  ), (1 2 3 4 5 6  )
\end{enumerate}
Invariant group of Orbit $5$:
\begin{enumerate}
\item Group Size is $2$.
\item Elements are : (1 2 3 5 4 6  ), (1 2 3 4 5 6  )
\end{enumerate}
Invariant group of Orbit $6$:
\begin{enumerate}
\item Group Size is $4$.
\item Elements are : (6 5 4 3 2 1  ), (6 4 5 2 3 1  ), (1 3 2 5 4 6  ), (1 2 3 4 5 6  )
\end{enumerate}
Invariant group of Orbit $7$:
\begin{enumerate}
\item Group Size is $2$.
\item Elements are : (5 3 2 6 1 4  ), (1 2 3 4 5 6  )
\end{enumerate}
Invariant group of Orbit $8$:
\begin{enumerate}
\item Group Size is $2$.
\item Elements are : (4 6 5 1 3 2  ), (1 2 3 4 5 6  )
\end{enumerate}
Invariant group of Orbit $9$:
\begin{enumerate}
\item Group Size is $4$.
\item Elements are : (5 3 2 6 1 4  ), (5 2 3 6 1 4  ), (1 3 2 4 5 6  ), (1 2 3 4 5 6  )
\end{enumerate}
Invariant group of Orbit $10$:
\begin{enumerate}
\item Group Size is $12$.
\item Elements are : (2 1 5 4 3 6  ), (1 2 5 4 3 6  ), (2 1 5 3 4 6  ), (1 2 5 3 4 6  ), (2 1 4 5 3 6  ), (1 2 4 5 3 6  ), (2 1 3 5 4 6  ), (1 2 3 5 4 6  ), (2 1 4 3 5 6  ), (1 2 4 3 5 6  ), (2 1 3 4 5 6  ), (1 2 3 4 5 6  )
\end{enumerate}
Invariant group of Orbit $11$:
\begin{enumerate}
\item Group Size is $2$.
\item Elements are : (1 2 3 5 4 6  ), (1 2 3 4 5 6  )
\end{enumerate}
Invariant group of Orbit $12$:
\begin{enumerate}
\item Group Size is $8$.
\item Elements are : (5 4 6 2 1 3  ), (5 4 6 1 2 3  ), (4 5 6 2 1 3  ), (4 5 6 1 2 3  ), (2 1 3 5 4 6  ), (1 2 3 5 4 6  ), (2 1 3 4 5 6  ), (1 2 3 4 5 6  )
\end{enumerate}
Invariant group of Orbit $14$:
\begin{enumerate}
\item Group Size is $2$.
\item Elements are : (2 1 4 3 5 6  ), (1 2 3 4 5 6  )
\end{enumerate}
Invariant group of Orbit $15$:
\begin{enumerate}
\item Group Size is $2$.
\item Elements are : (2 1 3 4 5 6  ), (1 2 3 4 5 6  )
\end{enumerate}
Invariant group of Orbit $16$:
\begin{enumerate}
\item Group Size is $8$.
\item Elements are : (4 3 2 1 6 5  ), (4 3 1 2 6 5  ), (3 4 2 1 6 5  ), (3 4 1 2 6 5  ), (2 1 4 3 5 6  ), (1 2 4 3 5 6  ), (2 1 3 4 5 6  ), (1 2 3 4 5 6  )
\end{enumerate}
\section{The two incidence matrix}
Incidence between Orbit of Extreme rays and Orbits of Facets
\begin{equation*}
\begin{array}{|c|cccccccccccccccc|c|}
\hline
&F1&F2&F3&F4&F5&F6&F7&F8&F9&F10&F11&F12&F13&F14&F15&F16&Inc\\
\hline
E1&11&18&60&96&96&48&80&104&44&16&88&28&144&72&72&16&993\\
E2&12&21&72&126&108&54&108&72&36&18&90&9&180&90&90&27&1113\\
\hline
\end{array}
\end{equation*}
Incidence between Orbits of Facets and Orbits of Extreme Rays
\begin{equation*}
\begin{array}{|c|cc|c|}
\hline
&E1&E2&Inc\\
\hline
F1&33&16&49\\
F2&27&14&41\\
F3&15&8&23\\
F4&12&7&19\\
F5&12&6&18\\
F6&12&6&18\\
F7&10&6&16\\
F8&13&4&17\\
F9&11&4&15\\
F10&12&6&18\\
F11&11&5&16\\
F12&14&2&16\\
F13&9&5&14\\
F14&9&5&14\\
F15&9&5&14\\
F16&8&6&14\\
\hline
\end{array}
\end{equation*}
\end{document}
