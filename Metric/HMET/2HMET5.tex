\documentclass[12pt]{article}
\usepackage{graphicx,amssymb,amsmath,vmargin,multicol}
\usepackage[english]{babel}
\title{The Cone 2HMET5}
\setpapersize{custom}{21cm}{29.7cm}
\setmarginsrb{1cm}{1cm}{1cm}{2cm}{0pt}{0pt}{0pt}{0pt}
\begin{document}
\maketitle
\section{The intrisic informations concerning the graph of extreme rays of 2HMET5.ext}
The graph considered is made of $37$ elements.\\
The group acting on 2HMET5.ext is Group2hmet5\\
Under the group action we have in fact only $3$ orbits to consider.\\
The diameter of the graph is $2$.\\
The list of orbits is with adjacency and Size info:
\begin{center}
\scriptsize
\begin{tabular}{ccccccccccc|c|c}
E1:&0&0&0&0&0&1&0&0&1&1&27&10\\
E2:&0&0&0&0&1&1&0&1&1&0&22&15\\
E3:&0&0&1&1&0&1&1&1&0&0&20&12\\
\end{tabular}
\end{center}
The representation matrix is:
\begin{center}
\scriptsize
\begin{tabular}{|c|ccc|c|c|}
\hline
&E1&E2&E3&Adj.&Size\\
\hline
E1& 9& 12& 6&27&10\\
E2& 8& 6& 8&22&15\\
E3& 5& 10& 5&20&12\\
\hline
Size&10&15&12&&37\\
\hline
\end{tabular}
\end{center}
\subsection{The local graphs}
The complemented local graph are:\\
We consider graph with cardinal inferior or equal to $20$\\
The complemented local graph for orbit $3$ is:
\begin{equation*}
\begin{array}{rrcl}
1&3&:&\,\,11\,\,12\,\,14\\
2&9&:&\,\,6\,\,7\,\,8\,\,10\,\,11\,\,13\,\,14\,\,18\,\,20\\
3&9&:&\,\,4\,\,6\,\,7\,\,8\,\,10\,\,12\,\,13\,\,14\,\,19\\
4&9&:&\,\,3\,\,8\,\,9\,\,11\,\,12\,\,13\,\,14\,\,18\,\,20\\
5&5&:&\,\,7\,\,8\,\,10\,\,17\,\,18\\
6&9&:&\,\,2\,\,3\,\,7\,\,9\,\,11\,\,12\,\,16\,\,18\,\,20\\
7&9&:&\,\,2\,\,3\,\,5\,\,6\,\,11\,\,14\,\,15\,\,17\,\,18\\
8&9&:&\,\,2\,\,3\,\,4\,\,5\,\,11\,\,12\,\,14\,\,15\,\,19\\
9&3&:&\,\,4\,\,6\,\,14\\
10&5&:&\,\,2\,\,3\,\,5\,\,18\,\,20\\
11&9&:&\,\,1\,\,2\,\,4\,\,6\,\,7\,\,8\,\,16\,\,17\,\,18\\
12&9&:&\,\,1\,\,3\,\,4\,\,6\,\,8\,\,14\,\,15\,\,17\,\,18\\
13&3&:&\,\,2\,\,3\,\,4\\
14&9&:&\,\,1\,\,2\,\,3\,\,4\,\,7\,\,8\,\,9\,\,12\,\,16\\
15&3&:&\,\,7\,\,8\,\,12\\
16&5&:&\,\,6\,\,11\,\,14\,\,17\,\,20\\
17&5&:&\,\,5\,\,7\,\,11\,\,12\,\,16\\
18&9&:&\,\,2\,\,4\,\,5\,\,6\,\,7\,\,10\,\,11\,\,12\,\,19\\
19&3&:&\,\,3\,\,8\,\,18\\
20&5&:&\,\,2\,\,4\,\,6\,\,10\,\,16\\
\end{array}
\end{equation*}
\subsection{scalar product between elements of orbits}
\noindent This subsection made sense only if the representation is orthogonal which is the usual case
Considering Orbit 1
\begin{enumerate}
\item Orbit Size is $10$.
\item Norm of elements is $3$.
\item Possible scalar product : $0$, $1$
\end{enumerate}
Considering Orbit 2
\begin{enumerate}
\item Orbit Size is $15$.
\item Norm of elements is $4$.
\item Possible scalar product : $1$, $2$
\end{enumerate}
Considering Orbit 3
\begin{enumerate}
\item Orbit Size is $12$.
\item Norm of elements is $5$.
\item Possible scalar product : $0$, $2$, $3$
\end{enumerate}
\subsection{invariant group of Orbits}
\noindent The invariant group of $x$ is the set of $g$ such that $g(x)=x$ denoted by $G_x$.\\
If $x$ and $y$ are in the same orbit then the groups $G_x$ and  $G_y$ are conjuguate.\\
For any element $x$ we denote $O_x$ its Orbit, we have $|O_x|\times |G_x|=|G|$
We print only nontrivial groups
Invariant group of Orbit $1$:
\begin{enumerate}
\item Group Size is $12$.
\item Elements are : (3 2 1 5 4  ), (3 1 2 5 4  ), (2 3 1 5 4  ), (1 3 2 5 4  ), (2 1 3 5 4  ), (1 2 3 5 4  ), (3 2 1 4 5  ), (3 1 2 4 5  ), (2 3 1 4 5  ), (1 3 2 4 5  ), (2 1 3 4 5  ), (1 2 3 4 5  )
\end{enumerate}
Invariant group of Orbit $2$:
\begin{enumerate}
\item Group Size is $8$.
\item Elements are : (4 3 2 1 5  ), (4 3 1 2 5  ), (3 4 2 1 5  ), (3 4 1 2 5  ), (2 1 4 3 5  ), (1 2 4 3 5  ), (2 1 3 4 5  ), (1 2 3 4 5  )
\end{enumerate}
Invariant group of Orbit $3$:
\begin{enumerate}
\item Group Size is $10$.
\item Elements are : (5 4 3 2 1  ), (5 3 4 1 2  ), (4 5 2 3 1  ), (3 5 1 4 2  ), (4 2 5 1 3  ), (3 1 5 2 4  ), (2 4 1 5 3  ), (1 3 2 5 4  ), (2 1 4 3 5  ), (1 2 3 4 5  )
\end{enumerate}
\section{The intrisic informations concerning the graph of facet rays of 2HMET5.ine}
The graph considered is made of $30$ elements.\\
The group acting on 2HMET5.ine is Group2hmet5\\
Under the group action we have in fact only $2$ orbits to consider.\\
The diameter of the graph is $2$.\\
The list of orbits is with adjacency and Size info:
\begin{center}
\scriptsize
\begin{tabular}{ccccccccccc|c|c}
F1:&-1&0&1&0&1&0&0&1&0&0&25&20\\
F2:&0&0&0&0&0&0&0&0&0&1&21&10\\
\end{tabular}
\end{center}
The representation matrix is:
\begin{center}
\scriptsize
\begin{tabular}{|c|cc|c|c|}
\hline
&F1&F2&Adj.&Size\\
\hline
F1& 16& 9&25&20\\
F2& 18& 3&21&10\\
\hline
Size&20&10&&30\\
\hline
\end{tabular}
\end{center}
No complemented local graph of size inferior to $20$.
\subsection{scalar product between elements of orbits}
\noindent This subsection made sense only if the representation is orthogonal which is the usual case
Considering Orbit 1
\begin{enumerate}
\item Orbit Size is $20$.
\item Norm of elements is $4$.
\item Possible scalar product : $-1$, $0$, $1$
\end{enumerate}
Considering Orbit 2
\begin{enumerate}
\item Orbit Size is $10$.
\item Norm of elements is $1$.
\item Possible scalar product : $0$
\end{enumerate}
\subsection{invariant group of Orbits}
\noindent The invariant group of $x$ is the set of $g$ such that $g(x)=x$ denoted by $G_x$.\\
If $x$ and $y$ are in the same orbit then the groups $G_x$ and  $G_y$ are conjuguate.\\
For any element $x$ we denote $O_x$ its Orbit, we have $|O_x|\times |G_x|=|G|$
We print only nontrivial groups
Invariant group of Orbit $1$:
\begin{enumerate}
\item Group Size is $6$.
\item Elements are : (3 2 1 4 5  ), (3 1 2 4 5  ), (2 3 1 4 5  ), (1 3 2 4 5  ), (2 1 3 4 5  ), (1 2 3 4 5  )
\end{enumerate}
Invariant group of Orbit $2$:
\begin{enumerate}
\item Group Size is $12$.
\item Elements are : (2 1 5 4 3  ), (1 2 5 4 3  ), (2 1 5 3 4  ), (1 2 5 3 4  ), (2 1 4 5 3  ), (1 2 4 5 3  ), (2 1 3 5 4  ), (1 2 3 5 4  ), (2 1 4 3 5  ), (1 2 4 3 5  ), (2 1 3 4 5  ), (1 2 3 4 5  )
\end{enumerate}
\section{The two incidence matrix}
Incidence between Orbit of Extreme rays and Orbits of Facets
\begin{equation*}
\begin{array}{|c|cc|c|}
\hline
&F1&F2&Inc\\
\hline
E1&14&7&21\\
E2&12&6&18\\
E3&10&5&15\\
\hline
\end{array}
\end{equation*}
Incidence between Orbits of Facets and Orbits of Extreme Rays
\begin{equation*}
\begin{array}{|c|ccc|c|}
\hline
&E1&E2&E3&Inc\\
\hline
F1&7&9&6&22\\
F2&7&9&6&22\\
\hline
\end{array}
\end{equation*}
\end{document}
