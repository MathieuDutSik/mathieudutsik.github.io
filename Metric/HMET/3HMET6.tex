\documentclass[12pt]{article}
\usepackage{graphicx,amssymb,amsmath,vmargin,multicol}
\usepackage[english]{babel}
\title{The Cone 3HMET6}
\setpapersize{custom}{21cm}{29.7cm}
\setmarginsrb{1cm}{1cm}{1cm}{2cm}{0pt}{0pt}{0pt}{0pt}
\begin{document}
\maketitle
\section{The intrisic informations concerning the graph of extreme rays of 3HMET6.ext}
The graph considered is made of $287$ elements.\\
The group acting on 3HMET6.ext is Group3hmet6\\
Under the group action we have in fact only $5$ orbits to consider.\\
The diameter of the graph is $3$.\\
The list of orbits is with adjacency and Size info:
\begin{center}
\scriptsize
\begin{tabular}{cccccccccccccccc|c|c}
E1:&0&0&0&0&0&0&0&0&0&1&0&0&0&1&1&136&20\\
E2:&0&0&0&0&0&0&0&0&1&1&0&0&1&1&0&94&45\\
E3:&0&0&0&0&0&1&0&1&0&1&0&1&1&0&0&70&72\\
E4:&0&0&0&0&1&1&1&0&1&0&1&1&0&0&0&54&60\\
E5:&0&0&1&0&1&1&1&0&0&0&1&0&0&2&1&26&90\\
\end{tabular}
\end{center}
The representation matrix is:
\begin{center}
\scriptsize
\begin{tabular}{|c|ccccc|c|c|}
\hline
&E1&E2&E3&E4&E5&Adj.&Size\\
\hline
E1& 19& 36& 36& 18& 27&136&20\\
E2& 16& 18& 24& 20& 16&94&45\\
E3& 10& 15& 20& 15& 10&70&72\\
E4& 6& 15& 18& 9& 6&54&60\\
E5& 6& 8& 8& 4& 0&26&90\\
\hline
Size&20&45&72&60&90&&287\\
\hline
\end{tabular}
\end{center}
No complemented local graph of size inferior to $20$.
\subsection{scalar product between elements of orbits}
\noindent This subsection made sense only if the representation is orthogonal which is the usual case
Considering Orbit 1
\begin{enumerate}
\item Orbit Size is $20$.
\item Norm of elements is $3$.
\item Possible scalar product : $0$, $1$
\end{enumerate}
Considering Orbit 2
\begin{enumerate}
\item Orbit Size is $45$.
\item Norm of elements is $4$.
\item Possible scalar product : $0$, $1$, $2$
\end{enumerate}
Considering Orbit 3
\begin{enumerate}
\item Orbit Size is $72$.
\item Norm of elements is $5$.
\item Possible scalar product : $0$, $1$, $2$, $3$
\end{enumerate}
Considering Orbit 4
\begin{enumerate}
\item Orbit Size is $60$.
\item Norm of elements is $6$.
\item Possible scalar product : $0$, $1$, $2$, $3$, $4$
\end{enumerate}
Considering Orbit 5
\begin{enumerate}
\item Orbit Size is $90$.
\item Norm of elements is $10$.
\item Possible scalar product : $2$, $4$, $6$
\end{enumerate}
\subsection{invariant group of Orbits}
\noindent The invariant group of $x$ is the set of $g$ such that $g(x)=x$ denoted by $G_x$.\\
If $x$ and $y$ are in the same orbit then the groups $G_x$ and  $G_y$ are conjuguate.\\
For any element $x$ we denote $O_x$ its Orbit, we have $|O_x|\times |G_x|=|G|$
We print only nontrivial groups
Invariant group of Orbit $1$:
\begin{enumerate}
\item Group Size is $36$.
\item Elements are : (3 2 1 6 5 4  ), (3 1 2 6 5 4  ), (2 3 1 6 5 4  ), (1 3 2 6 5 4  ), (2 1 3 6 5 4  ), (1 2 3 6 5 4  ), (3 2 1 6 4 5  ), (3 1 2 6 4 5  ), (2 3 1 6 4 5  ), (1 3 2 6 4 5  ), (2 1 3 6 4 5  ), (1 2 3 6 4 5  ), (3 2 1 5 6 4  ), (3 1 2 5 6 4  ), (2 3 1 5 6 4  ), (1 3 2 5 6 4  ), (2 1 3 5 6 4  ), (1 2 3 5 6 4  ), (3 2 1 4 6 5  ), (3 1 2 4 6 5  ), (2 3 1 4 6 5  ), (1 3 2 4 6 5  ), (2 1 3 4 6 5  ), (1 2 3 4 6 5  ), (3 2 1 5 4 6  ), (3 1 2 5 4 6  ), (2 3 1 5 4 6  ), (1 3 2 5 4 6  ), (2 1 3 5 4 6  ), (1 2 3 5 4 6  ), (3 2 1 4 5 6  ), (3 1 2 4 5 6  ), (2 3 1 4 5 6  ), (1 3 2 4 5 6  ), (2 1 3 4 5 6  ), (1 2 3 4 5 6  )
\end{enumerate}
Invariant group of Orbit $2$:
\begin{enumerate}
\item Group Size is $16$.
\item Elements are : (4 3 2 1 6 5  ), (4 3 1 2 6 5  ), (3 4 2 1 6 5  ), (3 4 1 2 6 5  ), (2 1 4 3 6 5  ), (1 2 4 3 6 5  ), (2 1 3 4 6 5  ), (1 2 3 4 6 5  ), (4 3 2 1 5 6  ), (4 3 1 2 5 6  ), (3 4 2 1 5 6  ), (3 4 1 2 5 6  ), (2 1 4 3 5 6  ), (1 2 4 3 5 6  ), (2 1 3 4 5 6  ), (1 2 3 4 5 6  )
\end{enumerate}
Invariant group of Orbit $3$:
\begin{enumerate}
\item Group Size is $10$.
\item Elements are : (5 4 3 2 1 6  ), (5 3 4 1 2 6  ), (4 5 2 3 1 6  ), (3 5 1 4 2 6  ), (4 2 5 1 3 6  ), (3 1 5 2 4 6  ), (2 4 1 5 3 6  ), (1 3 2 5 4 6  ), (2 1 4 3 5 6  ), (1 2 3 4 5 6  )
\end{enumerate}
Invariant group of Orbit $4$:
\begin{enumerate}
\item Group Size is $12$.
\item Elements are : (6 5 4 3 2 1  ), (6 4 5 3 1 2  ), (5 6 4 2 3 1  ), (4 6 5 1 3 2  ), (5 4 6 2 1 3  ), (4 5 6 1 2 3  ), (3 2 1 6 5 4  ), (3 1 2 6 4 5  ), (2 3 1 5 6 4  ), (1 3 2 4 6 5  ), (2 1 3 5 4 6  ), (1 2 3 4 5 6  )
\end{enumerate}
Invariant group of Orbit $5$:
\begin{enumerate}
\item Group Size is $8$.
\item Elements are : (1 6 3 5 4 2  ), (1 6 3 4 5 2  ), (3 5 1 6 2 4  ), (3 4 1 6 2 5  ), (3 5 1 2 6 4  ), (3 4 1 2 6 5  ), (1 2 3 5 4 6  ), (1 2 3 4 5 6  )
\end{enumerate}
\section{The intrisic informations concerning the graph of facet rays of 3HMET6.ine}
The graph considered is made of $45$ elements.\\
The group acting on 3HMET6.ine is Group3hmet6\\
Under the group action we have in fact only $2$ orbits to consider.\\
The diameter of the graph is $2$.\\
The list of orbits is with adjacency and Size info:
\begin{center}
\scriptsize
\begin{tabular}{cccccccccccccccc|c|c}
F1:&0&0&0&0&0&0&0&0&0&0&0&0&0&0&1&42&15\\
F2:&-1&0&1&0&1&0&0&1&0&0&0&1&0&0&0&39&30\\
\end{tabular}
\end{center}
The representation matrix is:
\begin{center}
\scriptsize
\begin{tabular}{|c|cc|c|c|}
\hline
&F1&F2&Adj.&Size\\
\hline
F1& 14& 28&42&15\\
F2& 14& 25&39&30\\
\hline
Size&15&30&&45\\
\hline
\end{tabular}
\end{center}
No complemented local graph of size inferior to $20$.
\subsection{scalar product between elements of orbits}
\noindent This subsection made sense only if the representation is orthogonal which is the usual case
Considering Orbit 1
\begin{enumerate}
\item Orbit Size is $15$.
\item Norm of elements is $1$.
\item Possible scalar product : $0$
\end{enumerate}
Considering Orbit 2
\begin{enumerate}
\item Orbit Size is $30$.
\item Norm of elements is $5$.
\item Possible scalar product : $-1$, $1$
\end{enumerate}
\subsection{invariant group of Orbits}
\noindent The invariant group of $x$ is the set of $g$ such that $g(x)=x$ denoted by $G_x$.\\
If $x$ and $y$ are in the same orbit then the groups $G_x$ and  $G_y$ are conjuguate.\\
For any element $x$ we denote $O_x$ its Orbit, we have $|O_x|\times |G_x|=|G|$
We print only nontrivial groups
Invariant group of Orbit $1$:
\begin{enumerate}
\item Group Size is $48$.
\item Elements are : (2 1 6 5 4 3  ), (1 2 6 5 4 3  ), (2 1 6 5 3 4  ), (1 2 6 5 3 4  ), (2 1 6 4 5 3  ), (1 2 6 4 5 3  ), (2 1 6 3 5 4  ), (1 2 6 3 5 4  ), (2 1 6 4 3 5  ), (1 2 6 4 3 5  ), (2 1 6 3 4 5  ), (1 2 6 3 4 5  ), (2 1 5 6 4 3  ), (1 2 5 6 4 3  ), (2 1 5 6 3 4  ), (1 2 5 6 3 4  ), (2 1 4 6 5 3  ), (1 2 4 6 5 3  ), (2 1 3 6 5 4  ), (1 2 3 6 5 4  ), (2 1 4 6 3 5  ), (1 2 4 6 3 5  ), (2 1 3 6 4 5  ), (1 2 3 6 4 5  ), (2 1 5 4 6 3  ), (1 2 5 4 6 3  ), (2 1 5 3 6 4  ), (1 2 5 3 6 4  ), (2 1 4 5 6 3  ), (1 2 4 5 6 3  ), (2 1 3 5 6 4  ), (1 2 3 5 6 4  ), (2 1 4 3 6 5  ), (1 2 4 3 6 5  ), (2 1 3 4 6 5  ), (1 2 3 4 6 5  ), (2 1 5 4 3 6  ), (1 2 5 4 3 6  ), (2 1 5 3 4 6  ), (1 2 5 3 4 6  ), (2 1 4 5 3 6  ), (1 2 4 5 3 6  ), (2 1 3 5 4 6  ), (1 2 3 5 4 6  ), (2 1 4 3 5 6  ), (1 2 4 3 5 6  ), (2 1 3 4 5 6  ), (1 2 3 4 5 6  )
\end{enumerate}
Invariant group of Orbit $2$:
\begin{enumerate}
\item Group Size is $24$.
\item Elements are : (4 3 2 1 5 6  ), (4 3 1 2 5 6  ), (4 2 3 1 5 6  ), (4 1 3 2 5 6  ), (4 2 1 3 5 6  ), (4 1 2 3 5 6  ), (3 4 2 1 5 6  ), (3 4 1 2 5 6  ), (2 4 3 1 5 6  ), (1 4 3 2 5 6  ), (2 4 1 3 5 6  ), (1 4 2 3 5 6  ), (3 2 4 1 5 6  ), (3 1 4 2 5 6  ), (2 3 4 1 5 6  ), (1 3 4 2 5 6  ), (2 1 4 3 5 6  ), (1 2 4 3 5 6  ), (3 2 1 4 5 6  ), (3 1 2 4 5 6  ), (2 3 1 4 5 6  ), (1 3 2 4 5 6  ), (2 1 3 4 5 6  ), (1 2 3 4 5 6  )
\end{enumerate}
\section{The two incidence matrix}
Incidence between Orbit of Extreme rays and Orbits of Facets
\begin{equation*}
\begin{array}{|c|cc|c|}
\hline
&F1&F2&Inc\\
\hline
E1&12&21&33\\
E2&11&18&29\\
E3&10&15&25\\
E4&9&12&21\\
E5&8&10&18\\
\hline
\end{array}
\end{equation*}
Incidence between Orbits of Facets and Orbits of Extreme Rays
\begin{equation*}
\begin{array}{|c|ccccc|c|}
\hline
&E1&E2&E3&E4&E5&Inc\\
\hline
F1&16&33&48&36&48&181\\
F2&14&27&36&24&30&131\\
\hline
\end{array}
\end{equation*}
\end{document}
